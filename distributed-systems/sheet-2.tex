\documentclass[10pt,a4paper]{article}
\usepackage[utf8]{inputenc}

% Math environments and macros
\usepackage{amsmath}
\usepackage{amsfonts}
\usepackage{amssymb}
\usepackage{amsthm}

% Define the page margin
\usepackage[margin=3cm]{geometry}

% Define \includegraphics to include graphics
\usepackage{graphicx}

% Better typography (font rendering)
\usepackage{microtype}

% Syntax highlighting
\usepackage{minted}

% Set global minted options
\setminted{linenos, autogobble, frame=lines, framesep=2mm}

\title{Distributed Systems, Sheet 2}
\author{Marten Lienen (03670270)}

\begin{document}

\maketitle

\section*{Exercise 1}

\subsection*{Part a)}

The third one because it has the lowest round trip time and thus the highest
accuracy.

\subsection*{Part b)}

\begin{equation*}
  10:54:28 + \frac{20}{2} = 10:54:38
\end{equation*}

\subsection*{Part c)}

Our new time can be at most 10 seconds off because we assume that both trips
take the same time. If that is true, our guessed time is totally
accurate. Otherwise we are $\pm 10$ seconds off in the worst case.

\subsection*{Part d)}

We would still use the third time, but under certain conditions we can improve
our guess of the RTT. Namely if we assume that the time is taken directly before
the response is sent, i.e. the known processing time is spent before the reading
of the clock, we can subtract the known processing time from our guess of the
RTT.
\begin{equation*}
  RTT = 20 - 8 = 12
\end{equation*}
So we would set our clock to $10:54:34$ with an accuracy of $\pm 6$ seconds.

\subsection*{Part e)}

No, this is impossible, because network latency and thus RTT cannot be zero and
the accuracy of Cristian's algorithm is lower bounded by $\frac{RTT}{2}$.

\section*{Exercise 2}

\subsection*{Part a)}

\begin{itemize}
\item The master actively synchronizes the time across all systems
\item Does not ensure the correct time but just that no single process deviates
  too far from the mean
\end{itemize}

\subsection*{Part b)}

We ignore the time from P1 because it has drifted too far from the its
P3's. Using the others we compute a mean time of
\begin{equation*}
  \frac{1}{4}
  \begin{pmatrix}
    1 & 1 & 1 & 1
  \end{pmatrix}
  \begin{pmatrix}
    8:44:52:874\\
    8:44:53:123\\
    8:44:53:100\\
    8:44:50:996
  \end{pmatrix}
  = 8:44:52:523
\end{equation*}

\subsection*{Part c)}

P3 will send the other processes the difference between their time and the mean,
i.e. the amount of time they have to adjust their clocks.
\begin{description}
\item[P1] $-3.612$
\item[P3] $-0.351$
\item[P4] $-0.577$
\item[P5] $+1.527$
\end{description}

\subsection*{Part d)}

We still kept a value that was skewed pretty far. As a result of this 4
processes had to slow down their clocks while only one process had to adjust it
forward.

\section*{Exercise 3}

\subsection*{Part a)}

\begin{description}
\item[Fault Tolerance] Every NTP server synchronizes with multiple others and
  recognizes and ignores ``insane'' time sources
\item[High Scalability] The hierarchichal architecture of NTP makes it highly
  scalable
\item[Availability] NTP copes with temporary network partitions
\end{description}

\subsection*{Part b)}

\subsubsection*{Symmetric}

\subsubsection*{Client/Server}

\subsubsection*{Broadcast}

\subsection*{Part c)}

\begin{equation*}
  \theta = \frac{(16:34:15.725 - 16:34:13.430) - (16:34:25.7 - 16:34:23.48)}{2} = 0.0375
\end{equation*}

\end{document}
