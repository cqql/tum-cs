\documentclass[10pt,a4paper]{article}
\usepackage[utf8]{inputenc}

% Define the page margin
\usepackage[margin=3cm]{geometry}

% Better typography (font rendering)
\usepackage{microtype}

% Math environments and macros
\usepackage{amsmath}
\usepackage{amsfonts}
\usepackage{amssymb}
\usepackage{amsthm}

% Define \includegraphics to include graphics
\usepackage{graphicx}

% Draw graphics from a text description
\usepackage{tikz}

% Syntax highlighting
\usepackage{minted}

% Set global minted options
\setminted{linenos, autogobble, frame=lines, framesep=2mm}

\usepackage{comment}

\usepackage{enumerate}

\title{Distributed Systems, Sheet 3}
\author{Marten Lienen (03670270)}

\begin{document}

\maketitle

\section*{Exercise 1}

\subsection*{Part a)}

% BEGIN RECEIVE ORGTBL logical-timestamps
\begin{tabular}{lr}
Event & Timestamp\\
\hline
a & 4\\
b & 3\\
c & 5\\
d & 6\\
e & 6\\
f & 8\\
\end{tabular}
% END RECEIVE ORGTBL logical-timestamps
\begin{comment}
#+ORGTBL: SEND logical-timestamps orgtbl-to-latex :splice nil :skip 0
| Event | Timestamp |
|-------+-----------|
| a     |         4 |
| b     |         3 |
| c     |         5 |
| d     |         6 |
| e     |         6 |
| f     |         8 |
\end{comment}

\subsection*{Part b)}

% BEGIN RECEIVE ORGTBL vector-timestamps
\begin{tabular}{ll}
Event & Timestamp\\
\hline
a & (3, 1, 0)\\
b & (3, 3, 0)\\
c & (2, 1, 2)\\
d & (2, 2, 3)\\
e & (3, 4, 2)\\
f & (4, 5, 2)\\
\end{tabular}
% END RECEIVE ORGTBL vector-timestamps
\begin{comment}
#+ORGTBL: SEND vector-timestamps orgtbl-to-latex :splice nil :skip 0
| Event | Timestamp |
|-------+-----------|
| a     | (3, 1, 0) |
| b     | (3, 3, 0) |
| c     | (2, 1, 2) |
| d     | (2, 2, 3) |
| e     | (3, 4, 2) |
| f     | (4, 5, 2) |
\end{comment}

\subsection*{Part c)}

\begin{enumerate}[i)]
\item $a \| c$
\item $b \| d$
\item $c \rightarrow e$
\item $c \rightarrow f$
\end{enumerate}

\subsection*{Part d)}

Yes, but only in the context of a single process.
If the events are from different processes, they are not comparable because logical clocks are only valid in the context of their own process.

\section*{Exercise 2}

I think this is impossible if there is a consumer of the incoming messages.
Suppose that processes 1 and 2 send the first messages at all at roughly the same size and process 3 receives the message from 2 first while process 4 receives the message from 1 first.
Then there is no way for 3 to know if it is supposed to wait for another message or not.

So I will assume that there is no consumer and messages remain in the queue forever.

\begin{minted}{python}
  # In Process i
  clock = [ 0 | 1..n ]

  def broadcastATO(msg):
      clock[i] = clock[i] + 1

      send(msg, clock, processId)

  messages = []
  def onReceiveATO(newMsg, newClock, newProcessId):
      clock = mergeVectorClocks(clock, newClock)
      clock[i] = clock[i] + 1

      for (msg, oldClock, processId) in messages:
          if happensBefore(oldClock, newClock):
              next
          else if happensBefore(newClock, oldClock) or newProcessId < processId:
              messages.insertBefore((newMsg, newClock, newProcessId), (msg, oldClock, processId))
              return

      messages.append((msg, newClock, processId))
\end{minted}

So we order incoming messages by their attached timestamps.
If two messages are concurrent we order them by their process IDs.

\section*{Exercise 3}

With the same argument as above I think that this is impossible.
A processor cannot know if it has to wait for a message without having received another message saying so.

\end{document}
