\documentclass[10pt,a4paper]{article}
\usepackage[utf8]{inputenc}

% Define the page margin
\usepackage[margin=3cm]{geometry}

% Better typography (font rendering)
\usepackage{microtype}

% Math environments and macros
\usepackage{amsmath}
\usepackage{amsfonts}
\usepackage{amssymb}
\usepackage{amsthm}

% Define \includegraphics to include graphics
\usepackage{graphicx}

% Draw graphics from a text description
\usepackage{tikz}

% Syntax highlighting
\usepackage{minted}

% Set global minted options
\setminted{linenos, autogobble, frame=lines, framesep=2mm}

% Import the comment environment for orgtbl-mode
\usepackage{comment}

% Do not indent paragraphs
\usepackage{parskip}

\title{Distributed Systems, Sheet 5}
\author{Marten Lienen (03670270)}

\begin{document}

\maketitle

\section*{Exercise 1}

\subsection*{Part a)}

A consistency model is a contract of a data store that specifies how writes are made visible between different clients, i.e. in which order and after how long a timespan.

\subsection*{Part b)}

A consistency model is needed to prove correctness of distributed programs.

\subsection*{Part c)}

Data-centric consistency concerns itself with consistency properties observed by multiple clients sharing a data store and sending concurrent read and write requests.
Client-centric consistency on the other hand is about consistency from the perspective of a single client without consideration of the concurrent actions of other clients.

\subsection*{Part d)}

\subsubsection*{Advantages}

\begin{itemize}
\item Replication improves read performance because data can be stored closer to the client
\item Replication guards you against data loss
\end{itemize}

\subsubsection*{Disadvantages}

\begin{itemize}
\item Multiple copies have to be kept consistent
\item You often need to have an additional mechanism to decide who owns a replicated resource
\end{itemize}

\subsection*{Part e)}

\begin{itemize}
\item CDNs replicate data to store it geographically close the client to minimize load times
\item Databases use replication to provide safety against crashes and data-loss
\end{itemize}

\section*{Exercise 2}

\subsection*{Part a)}

Yes, it is because every read returns the most recently written value.
A possible execution sequence would be $w(x)a\ r(x)a\ w(x)b\ r(x)b$.

\subsection*{Part b)}

It is not linearizable because $r(x)b$ might happen before $w(x)b$ which would therefore have to be $r(x)a$.
Furthermore it is impossible to be strictly consistent since strict consistency is a stronger requirement than linearizability.

If you only needed one valid sequence, it would be linearizable by the following sequence
\begin{equation*}
  w(x)a\ w(x)c\ w(x)b\ r(x)b\ r(x)b
\end{equation*}
However it would still not be strictly consistent because $r(x)b$ happens after $w(x)c$ in a global-time sense which is forbidden.

\subsection*{Part c)}

Figure 2.4 is sequentially consistent because all processes see the values of $x$ in the order $a, b, c$ and only after the corresponding writes happened.
However, it is not linearizable because linearizability postulates that all processes see a value after its write has definitely finished.
This is contradicted in this case by $P2$ which reads $r(x)b$ after $w(x)c$ happened.

\subsection*{Part d)}

Yes, it is causally consistent since all writes are concurrent and thus different processes reading different values is acceptable.
It is also sequentially consistent, because we cannot know in which order the processes saw the values.
Since time is not part of the postulates, it is totally possible that all processes see the values in the order $a, b, c$ but at different times and therefore read these three different values at the same point in time.

\subsection*{Part e)}

Figure 2.6 is causally consistent because the causality between the writes is $a -> b$ and $a -> c$ so reading $b$ and $c$ in different orders is fine as long as no process observes $a$ after $b$ or $c$.
However, it is not sequentially consistent, because P4 and P5 observe the values in a different order.

\subsection*{Part f)}

It is FIFO consistent with a possible execution sequence of
\begin{equation*}
w(x)a\ r(x)a\ r(x)a\ w(x)b\ r(x)b\ r(x)b\ w(x)d\ r(x)d\ w(x)c\ r(x)c
\end{equation*}

At the same time it also causally consistent, because the causalities are $a \rightarrow b \rightarrow c$ and $a \rightarrow b \rightarrow d$ which is compatible with the previously given execution sequence.

\section*{Exercise 3}

\subsection*{Part a)}

I do not see any problems here because the clients always reads the newest value.

\subsection*{Part b)}

The monotonic-write property is not kept up because $w(x2)$ happens before $w(x1)$ and $w(x3)$ happens before $w(x2)$.
All locations would have to write $w(x1)$ first before anything else and then $w(x2)$ before the client issues $w(x3)$.

\subsection*{Part c)}

Location 3 has not yet noticed $w(x1)$ but answered a subsequent read.
To be read-your-writes consistent, L3 would have to wait for $w(x1)$ before answering the client's request.

\subsection*{Part d)}

It may be considered writes-follow-reads consistent, because the writing of $w(x3)$ takes place on a more recent value, even though L3 has not yet received $w(x2)$ and is thus not read-your-writes consistent.

\end{document}
