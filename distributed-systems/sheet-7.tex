\documentclass[10pt,a4paper]{article}
\usepackage[utf8]{inputenc}

% Define the page margin
\usepackage[margin=3cm]{geometry}

% Better typography (font rendering)
\usepackage{microtype}

% Math environments and macros
\usepackage{amsmath}
\usepackage{amsfonts}
\usepackage{amssymb}
\usepackage{amsthm}

% Define \includegraphics to include graphics
\usepackage{graphicx}

% Draw graphics from a text description
\usepackage{tikz}

% Syntax highlighting
\usepackage{minted}

% Set global minted options
\setminted{linenos, autogobble, frame=lines, framesep=2mm}

% Import the comment environment for orgtbl-mode
\usepackage{comment}

% Do not indent paragraphs
\usepackage{parskip}

\title{Distributed Systems, Sheet 7}
\author{Marten Lienen (03670270)}

\begin{document}

\maketitle

\section*{Exercise 1}

\subsection*{Part a)}

A proxy is a server that forwards incoming requests in some manner.
A reverse proxy is the special case of a proxy that accepts requests from the outside and then forwards them to a predefined set of servers.
They are used for example for load balancing of web applications.

\subsection*{Part b)}

A proxy can serve as a cache and thus reduce communication with the proxied servers.
This is mostly used to proxy and cache web pages.

A reverse proxy can decide where to forward the request and is therefore often the right place to implement load-balancing.

\subsection*{Part c)}

\section*{Exercise 2}

Some requirements that I can think of are
\begin{itemize}
\item All caches have to agree what the current version of some cached data is
\item Content should be spread as evenly as possible across the caches
\item If a cache goes up or down, only little data should have to be moved
\end{itemize}

Traditional hashing violates the last requirement because it may remap a lot of data when the number of caches changes.

\subsection*{Part a)}

% BEGIN RECEIVE ORGTBL exercise-2-a
\begin{tabular}{rl}
Cache & Side IDs\\
\hline
0 & 2, 5, 8\\
1 & 0, 3, 6, 9\\
2 & 1, 4, 7\\
\end{tabular}
% END RECEIVE ORGTBL exercise-2-a
\begin{comment}
#+ORGTBL: SEND exercise-2-a orgtbl-to-latex :splice nil :skip 0
| Cache | Site IDs    |
|-------+------------|
|     0 | 2, 5, 8    |
|     1 | 0, 3, 6, 9 |
|     2 | 1, 4, 7    |
\end{comment}

\subsection*{Part b)}

% BEGIN RECEIVE ORGTBL exercise-2-b
\begin{tabular}{rl}
Cache & Site IDs\\
\hline
0 & 0, 4, 8\\
1 & 3, 7\\
2 & 2, 6\\
3 & 1, 5, 9\\
\end{tabular}
% END RECEIVE ORGTBL exercise-2-b
\begin{comment}
#+ORGTBL: SEND exercise-2-b orgtbl-to-latex :splice nil :skip 0
| Cache | Site IDs |
|-------+----------|
|     0 | 0, 4, 8  |
|     1 | 3, 7     |
|     2 | 2, 6     |
|     3 | 1, 5, 9  |
\end{comment}

% BEGIN RECEIVE ORGTBL exercise-2-b-moved
\begin{tabular}{rl}
Site & Moved?\\
\hline
0 & Y\\
1 & Y\\
2 & Y\\
3 & N\\
4 & Y\\
5 & Y\\
6 & Y\\
7 & Y\\
8 & N\\
9 & Y\\
\end{tabular}
% END RECEIVE ORGTBL exercise-2-b-moved
\begin{comment}
#+ORGTBL: SEND exercise-2-b-moved orgtbl-to-latex :splice nil :skip 0
| Site | Moved? |
|------+--------|
|    0 | Y      |
|    1 | Y      |
|    2 | Y      |
|    3 | N      |
|    4 | Y      |
|    5 | Y      |
|    6 | Y      |
|    7 | Y      |
|    8 | N      |
|    9 | Y      |
\end{comment}

So $80\%$ of the pages had to be moved.

\subsection*{Part c)}

% BEGIN RECEIVE ORGTBL exercise-2-c
\begin{tabular}{rl}
Cache & Site IDs\\
\hline
0 & 5, 8\\
1 & 0, 2, 3, 4, 6, 7, 9\\
2 & 1\\
\end{tabular}
% END RECEIVE ORGTBL exercise-2-c
\begin{comment}
#+ORGTBL: SEND exercise-2-c orgtbl-to-latex :splice nil :skip 0
| Cache | Site IDs            |
|-------+---------------------|
|     0 | 5, 8                |
|     1 | 0, 2, 3, 4, 6, 7, 9 |
|     2 | 1                   |
\end{comment}

\subsection*{Part d)}

% BEGIN RECEIVE ORGTBL exercise-2-d
\begin{tabular}{rl}
Cache & Site IDs\\
\hline
0 & 5, 8\\
1 & 0, 3, 6, 7\\
2 & 1\\
3 & 2, 4, 9\\
\end{tabular}
% END RECEIVE ORGTBL exercise-2-d
\begin{comment}
#+ORGTBL: SEND exercise-2-d orgtbl-to-latex :splice nil :skip 0
| Cache | Site IDs   |
|-------+------------|
|     0 | 5, 8       |
|     1 | 0, 3, 6, 7 |
|     2 | 1          |
|     3 | 2, 4, 9    |
\end{comment}

This time only $3 = 30\%$ of the sites were moved.

\subsection*{Part e)}

The mean loads in tasks c and d are
\begin{equation*}
  \bar{X}_{c} = \frac{10}{3} \approx 3.3 \qquad \bar{X}_{d} = \frac{10}{4} = 2.5
\end{equation*}

With these we can compute the standard deviations.
\begin{equation*}
  S_{c} = \sqrt{\frac{1}{2} \left( \left(\frac{4}{3}\right)^{2} + \left(\frac{11}{3}\right)^{2} + \left(\frac{7}{3}\right)^{2} \right)} = \sqrt{\frac{1}{2} \frac{16 + 121 + 49}{9}} = \sqrt{\frac{93}{9}} = \frac{\sqrt{93}}{3} \approx 3.2
\end{equation*}
\begin{equation*}
  S_{d} = \sqrt{\frac{1}{3} \left( 0.5^{2} + 1.5^{2} + 1.5^{2} + 0.5^{2} \right)} = \sqrt{\frac{5}{3}} \approx 1.3
\end{equation*}

Adding the extra cache reduced the standard deviation and thus improved the load of all caches.

\subsection*{Part f)}

First we sort the virtual caches by their hash values
% BEGIN RECEIVE ORGTBL exercise-2-f-sorted
\begin{tabular}{lr}
Hash & Virtual Cache\\
\hline
17D1 & 0-2\\
1FA3 & 2-1\\
5CBC & 1-1\\
8074 & 1-2\\
8010 & 2-2\\
8B02 & 0-1\\
\end{tabular}
% END RECEIVE ORGTBL exercise-2-f-sorted
\begin{comment}
#+ORGTBL: SEND exercise-2-f-sorted orgtbl-to-latex :splice nil :skip 0 :raw t
| Hash | Virtual Cache |
|------+---------------|
| 17D1 |           0-2 |
| 1FA3 |           2-1 |
| 5CBC |           1-1 |
| 8074 |           1-2 |
| 8010 |           2-2 |
| 8B02 |           0-1 |
\end{comment}

% BEGIN RECEIVE ORGTBL exercise-2-f
\begin{tabular}{rl}
Cache & Site IDs\\
\hline
0 & 0, 1, 3, 5, 6, 9\\
1 & 4\\
2 & 2, 7, 8\\
\end{tabular}
% END RECEIVE ORGTBL exercise-2-f
\begin{comment}
#+ORGTBL: SEND exercise-2-f orgtbl-to-latex :splice nil :skip 0
| Cache | Site IDs         |
|-------+------------------|
|     0 | 0, 1, 3, 5, 6, 9 |
|     1 | 4                |
|     2 | 2, 7, 8          |
\end{comment}

The mean load is $\frac{10}{3} \approx 3.3$.
The standard deviation is
\begin{equation*}
  S = \sqrt{\frac{1}{2} \left( \left(\frac{8}{3}\right)^{2} + \left( \frac{7}{3} \right)^{2} + \left( \frac{1}{3} \right)^{2} \right)} = \sqrt{\frac{1}{2} \frac{64 + 49 + 1}{9}} = \sqrt{\frac{57}{9}} = \frac{\sqrt{57}}{3} \approx 2.52
\end{equation*}

The distribution is about as good or bad as the original one and a lot worse than the one with four caches because the hash values are spread so unevenly.
Almost all the hash space is covered by cache 0.

\end{document}
