\documentclass[10pt,a4paper]{article}
\usepackage[utf8]{inputenc}

% Define the page margin
\usepackage[margin=3cm]{geometry}

% Better typography (font rendering)
\usepackage{microtype}

% Math environments and macros
\usepackage{amsmath}
\usepackage{amsfonts}
\usepackage{amssymb}
\usepackage{amsthm}

% Define \includegraphics to include graphics
\usepackage{graphicx}

% Draw graphics from a text description
\usepackage{tikz}

% Syntax highlighting
\usepackage{minted}

% Set global minted options
\setminted{linenos, autogobble, frame=lines, framesep=2mm}

% Import the comment environment for orgtbl-mode
\usepackage{comment}

% Do not indent paragraphs
\usepackage{parskip}

\title{Distributed Systems, Sheet 4}
\author{Marten Lienen (03670270)}

\begin{document}

\maketitle

\section*{Exercise 1}

\begin{description}
\item[Safety] Only one process can enter the CS at a time because a process can only get $n - 1$ acknowledgements if no other process is in the CS
\item[Liveness] You can prove inductively that all requests to enter the CS will succeed eventually if every process releases after a finite amount of time
\item[Fairness] A process cannot enter the CS twice while another process is waiting because the waiting process will be in the queues before the first one
\item[Order] This is not given.
  If node 1 and 2 both want to enter the CS and 2 sends the request first, 1 can still be the first to get access when it tries to enter the CS before receiving the request from 2.
  Because in this case 1 will queue the request from 2 while 2 will respond immediately.
  To resolve this you need to send timestamps with the requests as is done in the correct algorithm by Ricart \& Agrawala.
\end{description}

\section*{Exercise 2}

Processes are granted access to the CS according to their position on the ring instead of the order of requests.
So a simple example would be a ring of 3 processes where process 1 currently holds the token.
Now first process 3 tries to enter the CS, i.e. waits for the token, and then process 2.
However process 2 gets to enter first once process 1 releases because it is the next on the ring.

\section*{Exercise 3}

No, it cannot because you need some kind of circular dependency for a dead lock.
None the less you would have two totally separate lock systems for the two critical sections, so two state variables and therefore no interaction between the two locks/critical sections.

\section*{Exercise 4}

\subsection*{Part a)}

Nothing special because once a normal election starts you will have many elections of higher IDs in parallel anyway.

\subsection*{Part b)}

I think the bully algorithm handles the partition aspect already because it just elects a new leader once the current one becomes unavailable.
The interesting part is the reunification of the network.
For this case the leader could send a coordinator message periodically to ``convert'' peers once the network partitions are reunited.
Unless a leader with a higher ID receives such a coordinator message.
Then the higher leader would just respond with another one.

The only interesting case regarding a slow process is an election where the would-be leader is slow.
Here you could try the same treatment and just make everyone accept late or unexpected coordinator messages from peers with higher IDs.

\end{document}
