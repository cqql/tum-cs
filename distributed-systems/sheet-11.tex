\documentclass[10pt,a4paper]{article}
\usepackage[utf8]{inputenc}

% Define the page margin
\usepackage[margin=3cm]{geometry}

% Better typography (font rendering)
\usepackage{microtype}

% Math environments and macros
\usepackage{amsmath}
\usepackage{amsfonts}
\usepackage{amssymb}
\usepackage{amsthm}

% Define \includegraphics to include graphics
\usepackage{graphicx}

% Draw graphics from a text description
\usepackage{tikz}

% Syntax highlighting
\usepackage{minted}

% Set global minted options
\setminted{linenos, autogobble, frame=lines, framesep=2mm}

% Import the comment environment for orgtbl-mode
\usepackage{comment}

% Do not indent paragraphs
\usepackage{parskip}

\title{Distributed Systems, Sheet 11}
\author{Marten Lienen (03670270)}

\begin{document}

\maketitle

\section*{Exercise 1}

\subsection*{Part a)}

\subsubsection*{$0023$}

\begin{tabular}{c|cccc}
  Prefix & 0 & 1 & 2 & 3\\
  $\emptyset$ & \emph{0023} & 1002 & 2000 & \\
  $0$ & \emph{0023} & 0113 &  & 0322\\
  $00$ &  &  & \emph{0023} & \\
  $002$ &  &  &  & \emph{0023}\\
\end{tabular}

\begin{equation*}
  \{ 2210, 2231, 0113, 0133 \}
\end{equation*}

\subsubsection*{$1132$}

\begin{tabular}{c|cccc}
  Prefix & 0 & 1 & 2 & 3\\
  $\emptyset$ & 0322 & \emph{1132} & 2000 & \\
  $1$ & 1010 & \emph{1132} & 1223 & \\
  $11$ &  &  &  & \emph{1132}\\
  $113$ &  &  & \emph{1132} & \\
\end{tabular}

\begin{equation*}
  \{ 1002, 1010, 1223, 2000 \}
\end{equation*}

\subsubsection*{$2210$}

\begin{tabular}{c|cccc}
  Prefix & 0 & 1 & 2 & 3\\
  $\emptyset$ & 0322 & 1223 & \emph{2210} & \\
  $2$ & 2000 & 2112 & \emph{2210} & \\
  $22$ &  & \emph{2210} &  & 2231\\
  $221$ & \emph{2210} &  &  & \\
\end{tabular}

\begin{equation*}
  \{ 2000, 2112, 2231, 0023 \}
\end{equation*}

\subsection*{Part b)}

First $2210$ routes a join message to $1312$ to find the place for the new node.
This message is forwarded to $1223$ where it ends.
Then both $2210$ and $1223$ send their routing tables to the new nodes and $1312$ informs the nodes $1132$, $1223$, $2000$ and $2112$ of its arrival.
This amounts to a total of $2 + 2 + 4 = 8$ messages.

\subsection*{Part c)}

\begin{minted}{python}
  def route(orig, address, packet):
      dest = orig
      prefix = None

      for i in range(len(address)):
          next = routingTable(dest, prefix)[address[i]]
          prefix.append(address[i])

          if next is not None:
              dest = next


      deliver(dest, packet)
\end{minted}

The number of hops is upper bounded by the length of the addresses in base $B$.
At each step there are $B - 1 = 3$ choices that lead to a hop and $1$ that does not, so the expected number of hops in a full pastry network would be $B \cdot \frac{B - 1}{B} = B - 1 = 3$.

\subsection*{Part d)}

\begin{equation*}
  2210 -> 1132 -> 1002
\end{equation*}

\section*{Exercise 2}

\subsection*{Part a)}

\begin{tabular}{c|c|c}
  Round & Messages & TTL\\
  \hline
  1 & $1 \rightarrow 2$, $1 \rightarrow 3$ & 3\\
  2 & $2 \rightarrow 4$, $3 \rightarrow 4$, $3 \rightarrow 5$, $3 \rightarrow 6$ & 2\\
  3 & $4 \rightarrow \{ 6, 7 \}$, $5 \rightarrow \{ 6, 8 \}$, $6 \rightarrow \{ 4, 5, 7, 8 \}$ & 1 \\
  4 & $7 \rightarrow \{ 9, 10 \}$, $8 \rightarrow 11$ & 0
\end{tabular}

\subsection*{Part b)}

\begin{tabular}{c|c}
  Round & Messages\\
  \hline
  1 & $2 \rightarrow 1$, $5 \rightarrow 3$, $11 \rightarrow 8$\\
  2 & $3 \rightarrow 1$, $8 \rightarrow 6$\\
  3 & $6 \rightarrow 3$\\
  4 & $3 \rightarrow 1$
\end{tabular}

\subsection*{Part c)}

A lot of bandwidth would be wasted with resending messages.
Consider for example node $3$.
It first receives the message in round $1$ and forwards it.
But if it would not remember this, it would also receive and forward the request in round $2$ from peer $4$, in round $3$ from node $6$ and in round $4$ from node $5$.

Also it might not be possible to respond to the originator of the request, since the nodes do not store, where a request came from.

\section*{Exercise 3}

\subsection*{Part a)}

Rarest first guarantees that the diversity of packets among all peers is high and it therefore lowers the probability that a piece goes missing entirely because the last peer that had it stopped sharing.

The first piece is selected randomly, because in the beginning you just want to get some data to share as quickly as possible to be able to participate in the tit-for-tat game.

\subsection*{Part b)}

It is used here, so that a download is not blocked from finishing by a single piece that is downloaded from a slow peer.

It is not used for the entire download, because that would be mostly a waste of bandwidth and a slow transfer of a piece does not have bad consequences in the middle of a download as long as it still finishes at least one other packet.

\end{document}
