\documentclass[10pt,a4paper]{article}
\usepackage[utf8]{inputenc}

% Define the page margin
\usepackage[margin=3cm]{geometry}

% Better typography (font rendering)
\usepackage{microtype}

% Math environments and macros
\usepackage{amsmath}
\usepackage{amsfonts}
\usepackage{amssymb}
\usepackage{amsthm}

% Define \includegraphics to include graphics
\usepackage{graphicx}

% Draw graphics from a text description
\usepackage{tikz}

% Syntax highlighting
\usepackage{minted}

% Set global minted options
\setminted{linenos, autogobble, frame=lines, framesep=2mm}

% Import the comment environment for orgtbl-mode
\usepackage{comment}

% Do not indent paragraphs
\usepackage{parskip}

\title{Distributed Systems, Sheet 10}
\author{Marten Lienen (03670270)}

\begin{document}

\maketitle

\section*{Exercise 1}

\subsection*{Part a)}

In the worst case you need to make $n - 1$ hops, if the key falls in the key range of the predecessor.
If you also store a link to the predecessor, you need at most $\lfloor \frac{n}{2} \rfloor$ hops in case the key belongs to a node on the other side of the ring.

When a peer joins the ring you need to update the successor of the predecessor of the new peer and also the predecessor of the successor if we store predecessors too.
Additionally keys have to be transferred from the successor to the joined key.

Before leaving we have to do the inverse and update the successor/predecessor links again and transfer the keys back to the successor of the leaving node.

\subsection*{Part b)}

The distance between the node $i$ and position of the key is upper bounded by $2^{m}$.
Then we can take at most $\lceil\frac{2^{m}}{\lceil \sqrt{2^{m}} \rceil}\rceil$ jumps of length $\sqrt{2^{m}}$ to zone in on the target and then at most $\lceil \sqrt{2^{m}} \rceil$ further steps of length $1$ for fine-tuning.
This results in an upper bound of $2 \lceil \sqrt{2^{m}} \rceil$ jumps.

When a node joins the ring, you have to update at most $n$ skip pointers because all $n$ nodes could be right next to each other so that their skip pointers all point to the same node.
If you then place the new node ``in front'', all nodes have to update their skip pointers.

\subsection*{Part c)}

\begin{tabular}{c|c}
  Distance & Node\\
  \hline
  1 & $N_{6}$\\
  2 & $N_{6}$\\
  4 & $N_{10}$\\
  8 & $N_{15}$\\
  16 & $N_{22}$
\end{tabular}

No, that is impossible because there are 9 peers but the finger table only has length $m = 5$.

\subsection*{Part d)}

No, because there are other nodes between the ones in the finger table.

If the key is managed by the current node, you have found it.
Otherwise forward the request to the node in the node's fingertable that is closest to the position of the key but still in front of it.
If there is no such position, the key has to be managed by the next node, so forward the request to the next node.

Considering the example of looking up $K_{16}$ you first go to $K_{15}$ because $K_{16}$ is $13$ steps away from node $N_{3}$ and the largest distance smaller than $13$ in $N_{3}$'s finger table is $8$ which points to $K_{15}$.
Then $K_{15}$ looks in its fingertable for a distance of at most $1$ and forwards the request to $K_{17}$.
Then $K_{17}$ responds.

\subsection*{Part e)}

$N_{24}$ has to receive the keys from $N_{27}$ and all nodes that are less than half of the circle in front of $N_{24}$ have to potentially update their fingertables.

So $N_{10}$, $N_{15}$, $N_{17}$ and $N_{22}$ have to update their fingertables which costs up to $4 \cdot m$ requests because there are $4$ nodes and each fingertable has $m$ entries.

\subsection*{Part f)}

Move keys from $N_{0}$ to $N_{3}$ and update fingertables.
In particular you need to update the first three fingers of $N_{27}$, the fourth one of $N_{22}$ and the fifth of $N_{15}$.

You need ot update the fingertables of possibly all nodes and each one has $m$ entries.
So the operation requires at most $O(nm)$ lookups.

\subsection*{Part g)}

\section*{Exercise 2}

\subsection*{Part a)}

\begin{minted}{python}
  nearest = None
  distance = None

  for n in neighbours:
      dist = 0

      for j in range(1, d):
          dist = dist + (y[j] - (n.xmax[j] - n.xmin[j]) / 2 - n.xmin[j])**2

      if distance is None or dist < distance:
          nearest = n
          distance = dist

  forward(nearest, request)
\end{minted}

The request is forwarded at most $n$ times and at each step you have to compute the distance of all $2d$ neighbours to the target, which takes $d$ steps.
So it is in $O(nd^{2})$.

\subsection*{Part b)}

\subsection*{Part c)}

We split $P_{2}$ vertically and then all peers that will then be neighbours of $P_{4}$ have to update their neighbour set.

\end{document}
