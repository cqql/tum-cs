\documentclass[10pt,a4paper]{article}
\usepackage[utf8]{inputenc}

% Math environments and macros
\usepackage{amsmath}
\usepackage{amsfonts}
\usepackage{amssymb}
\usepackage{amsthm}

% Define the page margin
\usepackage[margin=3cm]{geometry}

% Define \includegraphics to include graphics
\usepackage{graphicx}

% Better typography (font rendering)
\usepackage{microtype}

% Syntax highlighting
\usepackage{minted}

% Set global minted options
\setminted{linenos, autogobble, frame=lines, framesep=2mm}

\title{Randomized Algorithms, Sheet 2}
\author{Marten Lienen (03670270)}

\begin{document}

\maketitle

\section*{Exercise 2.1}

\subsection*{Part a)}

Every edge contraction reduces the number of vertices by $1$ and we run the algorithm until only $2$ vertices are left.
This means that we perform $n - 2$ edge contractions per iteration of the algorithm and thus $2(n - 2)$ edge contractions in total.

In class we proved that the probability of discovering a min-cut is at least $\frac{2}{n^{2}}$.
Using this we can compute the probability of finding a min-cut in two iterations as the probability of not not finding one in any interation.
\begin{equation*}
  P \ge 1 - \left( 1 - \frac{2}{n^{2}} \right)^{2} = 1 - \frac{\left( n^{2} - 2 \right)^{2}}{n^{4}} = 1 - \frac{n^{4} - 4n^{2} + 4}{n^{4}} = 1 - 1 + \frac{4}{n^{2}} - \frac{4}{n^{4}} = \frac{4}{n^{2}} - \frac{4}{n^{4}}
\end{equation*}

\subsection*{Part b)}

The modified algorithm starts with $n - k$ edge contractions and then goes on to do $k - 2$ more contractions $l$ times.
So the total number is $n - k + l(k - 2)$.

Let $X$ be the event of picking one of the min-cut edges in the first $n - k$ contractions and thus destroying the min-cut.
Further let $Y_{i}$ be the event of finding a min-cut in the $i$th of the $l$ iterations.
Using this define $Y$ as the event of not finding the min-cut in any of the $l$ iterations, i.e. $Y = \cap_{i = 1}^{l} \lnot Y_{i}$.
Finally let $S$ be the event of success, namely that the modified algorithm finds a min-cut.
Then we can express the probability of $S$ as
\begin{align*}
  P[S] & = P[\lnot X] \cdot P[\lnot Y \mid \lnot X]\\
       & = P[\lnot X] \cdot P\left[ \lnot \left( \cap_{i = 1}^{l} \lnot Y_{i} \right) \mid \lnot X \right]\\
       & = P[\lnot X] \cdot \left( 1 - P\left[ \left( \cap_{i = 1}^{l} \lnot Y_{i} \right) \mid \lnot X \right] \right)\\
       & = P[\lnot X] \cdot \left( 1 - \prod_{i = 1}^{l} P\left[ \lnot Y_{i} \mid \lnot X \right] \right)\\
  \intertext{because the $Y_{i}$ are independent}
       & = P[\lnot X] \cdot \left( 1 - \left( P\left[ \lnot Y_{1} \mid \lnot X \right] \right)^{l} \right)\\
  \intertext{because the $Y_{i}$ are identically distributed}
\end{align*}

Now we can analyze the probability of not picking an edge of the min-cut in the first $n - k$ iterations.
Let $W_{i}$ be the event of picking a min-cut edge in the $i$th pick.
\begin{align*}
  P[\lnot X] & = P\left[ \cap_{i = 1}^{n - k} \lnot W_{i} \right]\\
             & = \prod_{i = 1}^{n - k} P\left[ \lnot W_{i} \mid \cap_{j = 1}^{i - 1} W_{j} \right]\\
             & \ge \prod_{i = 1}^{n - k} \left( 1 - \frac{2}{n - i + 1} \right)\\
  \intertext{As we have shown in the lecture}
             & = \prod_{i = 1}^{n - k} \left( \frac{n - i - 1}{n - i + 1} \right)\\
             & = \frac{(n - 2)(n - 3) \cdots (k - 1)}{n(n - 1) \cdots (k + 1)}\\
             & = \frac{\frac{(n - 2)!}{(k - 2)!}}{\frac{n!}{k!}}\\
             & = \frac{(n - 2)!k!}{n!(k - 2)!}\\
             & = \frac{k(k - 1)}{n(n - 1)}
\end{align*}

Next we will look for an upper bound of $P\left[ Y_{1} \mid \lnot X \right]$.
Here we can basically reuse our analysis done in the lecture respectively repeated in the previous paragraph.
Redefine $W_{i}$ for the last $k - 2$ iterations.
\begin{align*}
  P\left[ Y_{1} \mid \lnot X \right] & = P\left[ \cap_{i = 1}^{k - 2} \lnot W_{i} \right]\\
                                     & = \prod_{i = 1}^{k - 2} P\left[ \lnot W_{i} \mid \cap_{j = 1}^{i - 1} W_{j} \right]\\
                                     & \ge \prod_{i = 1}^{k - 2} \left( 1 - \frac{2}{k - i + 1} \right)\\
                                     & = \prod_{i = 1}^{k - 2} \left( \frac{k - i - 1}{k - i + 1} \right)\\
                                     & = \frac{(k - 2)(k - 3) \cdots 1}{k(k - 1) \cdots 3}\\
                                     & = \frac{(k - 2)!}{\frac{k!}{2!}}\\
                                     & = \frac{(k - 2)!2!}{k!}\\
                                     & = \frac{2}{k(k - 1)}
\end{align*}

Now we can go back to the beginning and plug these into our original equation.
\begin{align*}
  P[S] & = P[\lnot X] \cdot \left( 1 - \left( P\left[ \lnot Y_{1} \mid \lnot X \right] \right)^{l} \right)\\
       & = P[\lnot X] \cdot \left( 1 - \left( 1 - P\left[ Y_{1} \mid \lnot X \right] \right)^{l} \right)\\
       & \ge \frac{k(k - 1)}{n(n - 1)} \cdot \left( 1 - \left( 1 - \frac{2}{k(k - 1)} \right)^{l} \right)\\
       & = \frac{k(k - 1)}{n(n - 1)} \cdot \left( 1 - \left( \frac{k(k - 1) - 2}{k(k - 1)} \right)^{l} \right)\\
       & = \frac{k(k - 1)}{n(n - 1)} \cdot \left( \frac{(k(k - 1))^{l} - (k(k - 1) - 2)^{l}}{(k(k - 1))^{l}} \right)\\
       & = \frac{(k(k - 1))^{l} - (k(k - 1) - 2)^{l}}{n(n - 1)(k(k - 1))^{l - 1}}\\
\end{align*}

\section*{Exercise 2.2}

Let $n \in \mathbb{N}$, $V = \{ s, t, i_{1}, i_{n} \}$ and $E = \{ (s, i_{k}) \mid 1 \le k \le n \} \cup \{ (i_{k}, t) \mid 1 \le k \le n \}$.
Then the min-cut size of this graph is $n$ because you have to cut all $n$ pairs of $(s, i_{k}), (i_{k}, t)$ paths between $s$ and $t$.
The number of such cuts is $2^{n}$ because for each such pair you can either cut the former or the latter part and do so in all possible combinations.
Unfortunately this is not $2^{|V|}$ but at least it is of the right order of magnitude.

\section*{Exercise 2.3}

The exact statement canot be true because that would mean that every possible configuration of $U$ gives a cut size of at least $\frac{OPT}{2}$ which is easily disproven by considering $U = \{  \}$ in any graph with $OPT > 0$.
Therefore I guess that we were actually meant to examine the \emph{expected} size of the cut.

\begin{proof}
  \begin{align*}
    E[|\delta U|] & = E[|\{ uv \in E \mid u \in U, v \notin U \}|]\\
                  & = \sum_{uv \in E} P[(u \in U \land v \notin U) \lor (u \notin U \land v \in U)]\\
    \intertext{because the graph is undirected and thus the edges $uv$ and $vu$ are equal}
                  & = \sum_{uv \in E} \left( P[u \in U \land v \notin U] + P[u \notin U \land v \in U] \right)\\
    \intertext{because the two events are disjoint}
                  & = \sum_{uv \in E} \left( P[u \in U] \cdot P[v \notin U] + P[u \notin U] \cdot P[v \in U] \right)\\
    \intertext{because the decision about each vertex is independent of all other decisions}
                  & = \sum_{uv \in E} \left( \frac{1}{2} \cdot \frac{1}{2} + \frac{1}{2} \cdot \frac{1}{2} \right)\\
                  & = \sum_{uv \in E} \frac{1}{2} = \frac{1}{2} |E| \ge \frac{1}{2} OPT\\
    \intertext{because the size of the max cut cannot be greater than the number of edges, i.e. $|E| \ge OPT$}
  \end{align*}

  This proves that the expected size of the generated is cut is at least $\frac{1}{2}OPT$.
\end{proof}

\section*{Exercise 2.4}

\begin{minted}{c}
  struct LineSegment {
    // ...
  }

  struct Node {
    Node left;
    Node right;

    // If n > 0, the node is actually a leaf and left and right are empty
    // n is the number lines in this leaf
    int n;

    // The indices of the line segments in this leaf
    int[] lines;
  }

  RandAuto(S) {
    n = length(S)

    // Construct array of elements 1 to n
    // Runtime O(n)
    pi = 1..n

    // Generate a random permutation of the indices
    // Runtime O(n)
    pi = random_permutation(pi)

    // The root of the BPP tree
    tree = new Node(n = n, lines = pi)

    // Leaves that have more than 1 line segment
    leaves = new Queue(size = n)
    // The queue is implemented as a circular queue of fixed size so that #push
    // and #pop can be implemented in O(1)

    leaves.push(tree)

    // This loop will run once for every line segment and then once more for
    // every cut that happens during the execution of this algorithm. So it
    // runs O(n + X) times where X is the number of cuts occurring.
    while (!leaves.empty()) {
      // Everything in this loop besides the for loop has a run time of O(1)

      leaf = leaves.pop()

      if (leaf.n > 1) {
        // The first line segment index
        k = leaf.lines[0]
        splitter = S[k]

        leftN = 0
        rightN = 0
        leftLines = new Array(size = leaf.n)
        rightLines = new Array(size = leaf.n)

        // Run time O(leaf.n)
        for (i in leaf.lines) {
          line = S[i]

          if (i == k) {
            // We ignore the splitter itself and handle it at the end
          } else if (splitter.cuts(line)) {
            leftLines[leftN] = i
            rightLines[rightN] = i
            leftN++
            rightN++
          } else if (splitter.isLeftOf(line)) {
            rightLines[rightN] = i
            rightN++
          } else {
            leftLines[leftN] = i
            leftN++
          }
        }

        // Put the splitter on the side with less lines
        //
        // This behavior ensures that the algorithm terminates when we have only
        // two line segments left in the leaf.
        //
        // By putting it at the very end of the lines it will only be used for
        // splitting again after all other lines in the left half of the plane
        // have been used for splitting and only in the case that one of those
        // remaining lines has split the current splitter. That case is not
        // handled by the RandAuto algorithm as it was introduced in class.
        if (leftN < rightN) {
          leftLines[leftN] = k
          leftN++
        } else {
          rightLines[rightN] = k
          rightN++
        }

        left = new Node(n = leftN, lines = leftLines)
        right = new Node(n = rightN, lines = rightLines)

        leaf.n = 0
        leaf.lines = NULL
        leaf.left = left
        leaf.right = right

        leaves.push(leaf.left)
        leaves.push(leaf.right)
      }
    }

    return tree
  }
\end{minted}

The runtime of this implementation is clearly dominated by the loop in which we iterate over all the segments that yet need to be split.
From the analysis in the lecture we know that the expected number of splits and therefore segments respectively iterations is $O(n \log(n))$.
So we are iterating over all nodes of a random binary tree with $O(n \log(n))$ nodes and for each node perform equal to the number of children of that node because all its children have to be compared with the splitter chosen at that node.
So the run time of this implementation is the expected number of children in a random binary tree of size $O(n \log(n))$ times the number of nodes.
I guess it is $O(n \log(n) \log(n))$, but I cannot prove it.

\end{document}
