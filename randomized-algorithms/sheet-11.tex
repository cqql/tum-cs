\documentclass[10pt,a4paper]{article}
\usepackage[utf8]{inputenc}

% Define the page margin
\usepackage[margin=3cm]{geometry}

% Better typography (font rendering)
\usepackage{microtype}

% Math environments and macros
\usepackage{amsmath}
\usepackage{amsfonts}
\usepackage{amssymb}
\usepackage{amsthm}

% Define \includegraphics to include graphics
\usepackage{graphicx}

% Draw graphics from a text description
\usepackage{tikz}

% Syntax highlighting
\usepackage{minted}

% Set global minted options
\setminted{linenos, autogobble, frame=lines, framesep=2mm}

% Import the comment environment for orgtbl-mode
\usepackage{comment}

% Do not indent paragraphs
\usepackage{parskip}

\title{Randomized Algorithms, Sheet 11}
\author{Marten Lienen (03670270)}

\begin{document}

\maketitle

\section*{Exercise 11.1}

\begin{proof}
  Let $c$ be a node that can be written as the concatenation of two identical halves $c_{1}c_{1}$.
  There are $2^{\frac{n}{2}} = \sqrt{2^{n}} = \sqrt{N}$ nodes with addresses of the form $c_{1}d$ because there are $2^{\frac{n}{2}}$ bit strings $d$ of length $\frac{n}{2}$.
  All of these will send their packet to $dc_{1}$.
  The bit-fixing algorithm will route these packets over $c$ because after fixing the first $\frac{n}{2}$ bits, they arrive at $c_{1}c_{1}$.
  Therefore the last packet needs at least $\sqrt{N}$ rounds to pass this point.
\end{proof}

\section*{Exercise 11.2}

\begin{proof}
  Consider the permutation from exercise 1.
  In this case a packet is routed over $0_{n}$ iff at some intermediate step all its bits are zero, i.e. iff it does not have a $1$ that need not be fixed and all of its ones are fixed to zeros before any of its zeros are fixed to ones.

  Let $c$ be a random node and $c_{1}c_{2}$ its decomposition into two bitstrings of length $\frac{n}{2}$.
  The packet sent from $c$ does not have a $1$ that need not be fixed iff $c_{1}$ and $c_{2}$ do not have a $1$ at a common position $i$.
  Denote this event by $A$.
  \begin{equation*}
    P[A] = \prod_{i = 1}^{\frac{n}{2}} \left( 1 - \frac{1}{2} \cdot \frac{1}{2} \right) = \left( \frac{3}{4} \right)^{\frac{n}{2}}
  \end{equation*}

  Now let $c$ be such a packet, i.e. a packet in which every $1$ has to be fixed.
  What is the probability that all of its ones are fixed to zeros before any of its zeros are fixed to ones?
  Let $B$ denote this event, $p$ be the number of ones in $c$ and $q$ the total number of positions that need to be fixed.
  So the first $p$ fixes have to be from the positions where $c$ has ones.
  \begin{equation*}
    P[B | A, p, q] = \prod_{i = 1}^{p} \frac{p - (i - 1)}{q} = \frac{1}{q^{p}} \cdot p! \ge \frac{n!}{n^{n}}
  \end{equation*}
\end{proof}

\section*{Exercise 11.3}

\subsection*{Part a)}

\subsection*{Part b)}

\section*{Exercise 11.4}

We just round the values to the nearest integer where we round up for $\hat{x}_{i0}$ values and down for $\hat{x}_{i1}$ values.
Using this procedure you get $\bar{x}_{ij} \le 2\hat{x}_{ij}$ and from this it follows that
\begin{equation*}
  w_{s} = \sum_{i \in T_{b0}} \bar{x}_{i0} + \sum_{i \in T_{b1}} \bar{x}_{i1} \le \sum_{i \in T_{b0}} 2\hat{x}_{i0} + \sum_{i \in T_{b1}} 2\hat{x}_{i1} \le 2\hat{w} \le 2w_{0}
\end{equation*}

\end{document}
