\documentclass[10pt,a4paper]{article}
\usepackage[utf8]{inputenc}

% Define the page margin
\usepackage[margin=3cm]{geometry}

% Better typography (font rendering)
\usepackage{microtype}

% Math environments and macros
\usepackage{amsmath}
\usepackage{amsfonts}
\usepackage{amssymb}
\usepackage{amsthm}

% Define \includegraphics to include graphics
\usepackage{graphicx}

% Draw graphics from a text description
\usepackage{tikz}

% Syntax highlighting
\usepackage{minted}

% Set global minted options
\setminted{linenos, autogobble, frame=lines, framesep=2mm}

\title{Virtual Physics, Sheet 3}
\author{Marten Lienen (03670270)}

\begin{document}

\maketitle

\section*{Exercise A}

\section*{Exercise B}

If the voltage applied to the diode is negative, i.e. the current is conducted backwards, the diode blocks all current.
This means that $I = 0$ for $U < 0$.
If the current is applied in the direction of the diode with a positive voltage, it conducts everything.
Written down it is $I = \infty$ if $U \ge 0$.

Cable ties are a mechanical analogon.
They allow movement in one direction but block it in the other.

\section*{Exercise C}

See the article on hydraulic analogy on wikipedia.

\subsection*{Resistor}

A constrictive pipe that requires more pressure to move the same amout of fluid through.

\subsection*{Capacitor}

A tube with an elastic barrier that deforms and pushes water out when the incoming pressure is increased.

\subsection*{Coil}

A massive paddle wheel with a lot of inertia placed in the stream.
The pressure will slowly make the wheel move which will then keep moving for a while even after the pressure has dropped again.

\subsection*{Transformer}

\end{document}
