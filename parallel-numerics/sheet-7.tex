\documentclass[10pt,a4paper]{article}
\usepackage[utf8]{inputenc}

% Define the page margin
\usepackage[margin=3cm]{geometry}

% Better typography (font rendering)
\usepackage{microtype}

% Math environments and macros
\usepackage{amsmath}
\usepackage{amsfonts}
\usepackage{amssymb}
\usepackage{amsthm}

% Define \includegraphics to include graphics
\usepackage{graphicx}

% Syntax highlighting
\usepackage{minted}

% Set global minted options
\setminted{linenos, autogobble, frame=lines, framesep=2mm}

% Import the comment environment for orgtbl-mode
\usepackage{comment}

% Do not indent paragraphs
\usepackage{parskip}

\title{Parallel Numerics, Sheet 7}
\author{Marten Lienen (03670270)}

\begin{document}

\maketitle

\section*{Exercise 1}

\subsection*{Part a)}

Hypercube, cartesian grid, tree, fat tree, ring.

You can use virtual topologies to map MPI ranks to nodes such that they are optimized for a ring communication pattern for example.

\subsection*{Part b)}

A two-dimensional cartesian grid with wrap around.

\subsection*{Part c)}

\subsection*{Part d)}

Restrict messages to subgroups of processors, for example the second row of a two-dimensional grid topology.

\section*{Exercise 2}

\subsection*{Part a)}

Elements of the same color can be computed in parallel and have dependency trees of the same depth.

\subsection*{Part b)}

Chess board coloring.

\subsection*{Part c)}

\section*{Exercise 3}

In the general case of a random matrix CSR and CSC perform best.
In practice there are probably scenario where for example the diagonal-wise storage would perform best.

\begin{figure}[h]
  \centering
  \begin{minted}{sh}
    $ ./a.out
    Full: 0.104327s
    Coordinate Form: 0.020137s
    Compressed Sparse Row: 0.016188s
    Compressed Sparse Column: 0.016050s
    CSR with Extracted Diagonal: 0.016366s
    Diagonal-Wise Storage: 0.135155s
    Rectangular Row Storage: 0.018884s
    Jagged Diagonal: 0.026019s
  \end{minted}
  \caption{Run with a random matrix $n = 4000$ and a mean fill-in of $10\%$}
  \label{fig:}
\end{figure}

\end{document}
