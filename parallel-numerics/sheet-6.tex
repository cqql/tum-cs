\documentclass[10pt,a4paper]{article}
\usepackage[utf8]{inputenc}

% Define the page margin
\usepackage[margin=3cm]{geometry}

% Better typography (font rendering)
\usepackage{microtype}

% Math environments and macros
\usepackage{amsmath}
\usepackage{amsfonts}
\usepackage{amssymb}
\usepackage{amsthm}

% Define \includegraphics to include graphics
\usepackage{graphicx}

% Syntax highlighting
\usepackage{minted}

% Set global minted options
\setminted{linenos, autogobble, frame=lines, framesep=2mm}

% Import the comment environment for orgtbl-mode
\usepackage{comment}

% Do not indent paragraphs
\usepackage{parskip}

\title{Parallel Numerics, Sheet 6}
\author{Marten Lienen (03670270)}

\begin{document}

\maketitle

\section*{Exercise 1}

\subsection*{Part a)}

Each node has to pass the ratio $\frac{a_{k + 1}}{b_{k}}$ to its right neighbour who only then can update its own $b_{k}$ to continue the chain.
Therefore there is only one busy node at any point in time.

\subsection*{Part b)}

The first step fills-in the left border cells of the matrix block while at the same time eliminating the subdiagonal.
Second, the cells second from the right border are filled-in while the superdiagonal is eliminated.
Later steps only manipulate and eliminate existing cells.

In the end this algorithm can run in place because for every fill-in there is a cell eliminated at the same time.

\subsection*{Part c)}

\section*{Exercise 2}

\subsection*{Part a)}

\mintinline{c}{u = 1, v = 2, w = undefined} and blocking until another message with tag $0$ is received.

\subsection*{Part b)}

\mintinline{c}{u = 1, v = 3, w = 2}

\section*{Exercise 3}

You relate the $x_{i}$ to variables that are further and further away until they drop off the edge of the matrix and you get an equation like $b_{i}^{(N)}x_{i} = d_{i}^{(N)}$.

There is no fill-in generated and it is easy to parallelize.
Every core computes a block of the $x_{i}$ and then shares only the necessary updates with the respective nodes.

\end{document}
