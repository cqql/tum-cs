\documentclass[10pt,a4paper]{article}
\usepackage[utf8]{inputenc}

% Define the page margin
\usepackage[margin=3cm]{geometry}

% Better typography (font rendering)
\usepackage{microtype}

% Math environments and macros
\usepackage{amsmath}
\usepackage{amsfonts}
\usepackage{amssymb}
\usepackage{amsthm}

% Define \includegraphics to include graphics
\usepackage{graphicx}

% Draw graphics from a text description
\usepackage{tikz}

% Syntax highlighting
\usepackage{minted}

% Set global minted options
\setminted{linenos, autogobble, frame=lines, framesep=2mm}

% Import the comment environment for orgtbl-mode
\usepackage{comment}

% Do not indent paragraphs
\usepackage{parskip}

\title{Parallel Numerics, Sheet 2}
\author{Marten Lienen (03670270)}

\begin{document}

\maketitle

\section*{Exercise 1}

\subsection*{Part a)}

\subsection*{Part b)}

% BEGIN RECEIVE ORGTBL exercise-1-b
\begin{tabular}{rrrr}
Procs/n & 2 & 4 & 8\\
\hline
1 & 3 & 3.75 & 3.9375\\
4 & 3.9375 & 3.984375 & 3.99609375\\
16 & 3.99609375 & 3.9990234375 & 3.9997558594\\
\end{tabular}
% END RECEIVE ORGTBL exercise-1-b
\begin{comment}
#+ORGTBL: SEND exercise-1-b orgtbl-to-latex :splice nil :skip 0
| Procs/n | 2            | 4              | 8              |
|---------+--------------+----------------+----------------|
|       1 | $3$          | $3.75$         | $3.9375$       |
|       4 | $3.9375$     | $3.984375$     | $3.99609375$   |
|      16 | $3.99609375$ | $3.9990234375$ | $3.9997558594$ |
\end{comment}

\subsection*{Part c)}

Analytical solution
\begin{equation*}
  \int_{0}^{2\pi} x \sin(10x) ~\mathrm{d}x = -\left[ \frac{x}{10}\cos(10x) + \frac{1}{100} \sin(10x) \right]_{0}^{2\pi} = - \left( \frac{2\pi}{10}\cos(20\pi) + \frac{1}{100}\sin(20\pi) \right) = - \frac{2\pi}{10} \approx -0.62832
\end{equation*}

% BEGIN RECEIVE ORGTBL exercise-1-c
\begin{tabular}{rrrr}
Procs/n & 2 & 4 & 8\\
\hline
1 & 0 & 0 & -2.4674011003\\
4 & -2.4674011003 & 0.5110154998 & -0.4121661766\\
16 & -0.4121661766 & -0.5770228472 & -0.6156510848\\
\end{tabular}
% END RECEIVE ORGTBL exercise-1-c
\begin{comment}
#+ORGTBL: SEND exercise-1-c orgtbl-to-latex :splice nil :skip 0
| Procs/n | 2               | 4               | 8               |
|---------+-----------------+-----------------+-----------------|
|       1 | $0$             | $0$             | $-2.4674011003$ |
|       4 | $-2.4674011003$ | $0.5110154998$  | $-0.4121661766$ |
|      16 | $-0.4121661766$ | $-0.5770228472$ | $-0.6156510848$ |
\end{comment}
The numbers oscillate because a basic digital signal processing theorem says that you need to probe a function with at least double the highest frequency on the interval.
The frequency on the interval $[0, 2\pi]$ is $10$ and it is true that from $20$ probes the numbers stop to oscillate and converge towards the true value.

\subsection*{Part d)}

The node that integrates over the discontinuity performs an ever increasing number of iterations while every other node terminates in one or two iterations.

\subsection*{Part e)}

It is called reducing.

\section*{Exercise 2}

\subsection*{Part a)}

\subsection*{Part b)}

$b_{0}$ could have any of the first two values, $b_{3}$ any the last two and $b_{1}, b_{2}$ any of the four values.
The reason is that any interleaving of messages where the messages from each sender stay in order is a valid outcome.

\subsection*{Part c)}

A blocking MPI command halts program execution until the used resources, e.g. a message buffer, can be reused again.
This does not necessarily mean that for example a message was really received by the intended recipient.

\subsection*{Part d)}

MPI communication commands are not synchronized which means that its termination does not guarantee reception of a message.
To guarantee this you need to draw on special synchronization operations like $\mathtt{MPI\_Barrier}$.

\section*{Exercise 3}

MPI has datatypes for all the standard number types in C, i.e. short, int, float, etc.
Furthermore there are types for some composite datatypes, e.g. a struct of two ints.

\end{document}
