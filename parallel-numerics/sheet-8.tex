\documentclass[10pt,a4paper]{article}
\usepackage[utf8]{inputenc}

% Define the page margin
\usepackage[margin=3cm]{geometry}

% Better typography (font rendering)
\usepackage{microtype}

% Math environments and macros
\usepackage{amsmath}
\usepackage{amsfonts}
\usepackage{amssymb}
\usepackage{amsthm}

% Define \includegraphics to include graphics
\usepackage{graphicx}

% Syntax highlighting
\usepackage{minted}

% Set global minted options
\setminted{linenos, autogobble, frame=lines, framesep=2mm}

% Import the comment environment for orgtbl-mode
\usepackage{comment}

% Do not indent paragraphs
\usepackage{parskip}

\title{Parallel Numerics, Sheet 8}
\author{Marten Lienen (03670270)}

\begin{document}

\maketitle

\section*{Exercise 1}

Let $A = L + D + U$ where $L$ is strictly lower triangular, $D$ is diagonal and $U$ is strictly upper triangular.

\subsection*{Part a)}

\subsubsection*{Richardson}

\begin{equation*}
  x^{(n + 1)} = (I - A)x^{(n)} + b
\end{equation*}
\begin{equation*}
  M = I \qquad N = I - A
\end{equation*}

\subsubsection*{Jacobi}

\begin{equation*}
  M = D \qquad N = -(L + U)
\end{equation*}
\begin{equation*}
  Dx^{(n + 1)} = -(L + U)x^{(n)} + b
\end{equation*}

\subsubsection*{Gauß-Seidel}

\begin{equation*}
  M = L + D \qquad N = -U
\end{equation*}
\begin{equation*}
  (L + D)x^{(n + 1)} = -Ux^{(n)} + b
\end{equation*}

\subsection*{Part b)}

\subsubsection*{Richardson}

\begin{minted}{python}
  for t in range(1, ..):
      # SAXPY
      for i in range(1, n):
          x[t, i] = x[t - 1, i] + b[i]

      # GAXPY
      for i in range(1, n):
          for j in range(1, n):
               x[t, i] -= A[i, j] * x[t - 1, j]
\end{minted}

\subsubsection*{Jacobi}

\subsubsection*{Gauß-Seidel}

\subsection*{Part c)}

\subsection*{$\omega$-Jacobi}

\begin{align*}
  x^{(i + 1)} & = \omega(D^{-1}(L + U)x^{(i)} + D^{-1}b) + (1 - \omega)x^{(i)}\\
              & = (1 - \omega)x^{(i)} + \omega D^{-1}(L + U)x^{(i)} + \omega D^{-1}b\\
              & = ((1 - \omega)I + \omega D^{-1}(L + U))x^{(i)} + \omega D^{-1}b
\end{align*}

\subsection*{SOR}

\begin{align*}
  x^{(i + 1)} & = \omega(-(L + D)^{-1}Ux^{(i)} + (L + D)^{-1}b) + (1 - \omega)x^{(i)}\\
              & = (1 - \omega)x^{(i)} - \omega(L + D)^{-1}Ux^{(i)} + \omega(L + D)^{-1}b\\
              & = ((1 - \omega)I - \omega(L + D)^{-1}U)x^{(i)} + \omega(L + D)^{-1}b
\end{align*}
% Not correct

\section*{Exercise 2}

\subsection*{Part a)}

\begin{equation*}
  r^{(n)} = b - Ax^{(n)} = Ax - Ax^{(n)} = A(x - x^{(n)}) \Leftrightarrow x^{(n + 1)} = x^{(n)} + A^{-1}r^{(n)}
\end{equation*}

\subsubsection*{Richardson}

\subsubsection*{Jacobi}

\subsubsection*{Gauß-Seidel}

\subsection*{Part b)}

\subsection*{Part c)}

\subsection*{Part d)}

\section*{Exercise 3}

Das Ergebnis ist am Ende in Teilstücken auf die Nodes verteilt.

% BEGIN RECEIVE ORGTBL exercise-3
\begin{tabular}{rr}
Number of Processors & Time\\
\hline
1 & 3.3\\
2 & 2.0\\
4 & 1.9\\
\end{tabular}
% END RECEIVE ORGTBL exercise-3
\begin{comment}
#+ORGTBL: SEND exercise-3 orgtbl-to-latex :splice nil :skip 0
| Number of Processors | Time |
|----------------------+------|
|                    1 |  3.3 |
|                    2 |  2.0 |
|                    4 |  1.9 |
\end{comment}


\end{document}
