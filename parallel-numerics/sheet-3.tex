\documentclass[10pt,a4paper]{article}
\usepackage[utf8]{inputenc}

% Define the page margin
\usepackage[margin=3cm]{geometry}

% Better typography (font rendering)
\usepackage{microtype}

% Math environments and macros
\usepackage{amsmath}
\usepackage{amsfonts}
\usepackage{amssymb}
\usepackage{amsthm}

% Define \includegraphics to include graphics
\usepackage{graphicx}

% Syntax highlighting
\usepackage{minted}

% Set global minted options
\setminted{linenos, autogobble, frame=lines, framesep=2mm}

% Import the comment environment for orgtbl-mode
\usepackage{comment}

% Do not indent paragraphs
\usepackage{parskip}

\title{Parallel Numerics, Sheet 3}
\author{Marten Lienen (03670270)}

\begin{document}

\maketitle

\section*{Exercise 1}

\subsection*{Part a)}

\subsection*{Part b)}

Such an algorithm always has to perform at least $2^{k} - 1$ additions since otherwise there would be a number missing from the sum.
Therefore we can just add up all the numbers from first to last.

\subsection*{Part c)}

Do it as illustrated in the scheme.
This scheme finishes in $k$ timesteps.

You can use at most $\frac{2^{k}}{2}$ processors where each of them sums two of the values in the first step.

\section*{Exercise 2}

% BEGIN RECEIVE ORGTBL exercise-2
\begin{tabular}{rlll}
BLAS level & description & complexity & example\\
\hline
1 & vector-vector operations & $n$ & $w \leftarrow \alpha v + w$\\
2 & matrix-vector operations & $n^{2}$ & $w \leftarrow \alpha Av \beta w$\\
3 & matrix-matrix operations & $n^{3}$ & $C \leftarrow  \alpha AB + \beta C$\\
\end{tabular}
% END RECEIVE ORGTBL exercise-2
\begin{comment}
#+ORGTBL: SEND exercise-2 orgtbl-to-latex :splice nil :skip 0 :raw t
| BLAS level | description              | complexity | example                             |
|------------+--------------------------+------------+-------------------------------------|
|          1 | vector-vector operations | $n$        | $w \leftarrow \alpha v + w$         |
|          2 | matrix-vector operations | $n^{2}$    | $w \leftarrow \alpha Av + \beta w$     |
|          3 | matrix-matrix operations | $n^{3}$    | $C \leftarrow  \alpha AB + \beta C$ |
\end{comment}
\\
BLAS routines are dynamically linked because they are often fitted to the very architecture and therefore may differ from node to node.

\section*{Exercise 3}

\subsection*{Part a)}

\subsection*{Part b)}

After the termination of a blocking MPI command you are free to reuse the message buffers whereas you have to explicitly \mintinline{c}{MPI_Wait} somewhen in the case of non-blocking communication.

\subsection*{Part c)}

Non-blocking commands let you hide latency between nodes by filling the waiting time with computation.

\section*{Exercise 4}

\subsection*{Part a)}

% BEGIN RECEIVE ORGTBL exercise-4-a
\begin{tabular}{rr}
Proc Rank & a\\
\hline
0 & 6\\
1 & 2\\
2 & 3\\
\end{tabular}
% END RECEIVE ORGTBL exercise-4-a
\begin{comment}
#+ORGTBL: SEND exercise-4-a orgtbl-to-latex :splice nil :skip 0
| Proc Rank | a |
|-----------+---|
|         0 | 6 |
|         1 | 2 |
|         2 | 3 |
\end{comment}

\subsection*{Part b)}

Sum reduction.

\subsection*{Part c)}

On the root node the addition \mintinline{c}{a += b} would possibly be executed before the value of \mintinline{c}{b} was actually received.
Therefore you would add an undefined value to \mintinline{c}{a}.
Furthermore it is totally possible that not a single value is received before the whole loop finishes.

\section*{Exercise 5}

\subsection*{Vector Addition}

% BEGIN RECEIVE ORGTBL exercise-5-sum
\begin{tabular}{rlrr}
Procs & Time & Speedup & Efficiency\\
\hline
1 & 0.040754s & 1 & 1\\
2 & 0.021910s & 1.86 & 0.93\\
4 & 0.016606s & 2.45 & 0.49\\
8 & 0.007573s & 5.38 & 0.67\\
\end{tabular}
% END RECEIVE ORGTBL exercise-5-sum
\begin{comment}
#+ORGTBL: SEND exercise-5-sum orgtbl-to-latex :splice nil :skip 0
| Procs | Time      | Speedup | Efficiency |
|-------+-----------+---------+------------|
|     1 | 0.040754s |       1 |          1 |
|     2 | 0.021910s |    1.86 |       0.93 |
|     4 | 0.016606s |    2.45 |       0.49 |
|     8 | 0.007573s |    5.38 |       0.67 |
\end{comment}

\subsection*{Inner Product}

You could parallelize the final sum by SUM reducing the partial results instead of collecting them all on the root node.
This, however, is unnecessary since the number of processors is much smaller than the number of vector entries.

% BEGIN RECEIVE ORGTBL exercise-5-inner
\begin{tabular}{rlrr}
Procs & Time & Speedup & Efficiency\\
\hline
1 & 0.033900s & 1 & 1\\
2 & 0.018552s & 1.83 & 0.915\\
4 & 0.009315s & 3.64 & 0.91\\
8 & 0.004800s & 7.06 & 0.883\\
\end{tabular}
% END RECEIVE ORGTBL exercise-5-inner
\begin{comment}
#+ORGTBL: SEND exercise-5-inner orgtbl-to-latex :splice nil :skip 0
| Procs | Time      | Speedup | Efficiency |
|-------+-----------+---------+------------|
|     1 | 0.033900s |       1 |          1 |
|     2 | 0.018552s |    1.83 |      0.915 |
|     4 | 0.009315s |    3.64 |       0.91 |
|     8 | 0.004800s |    7.06 |      0.883 |
\end{comment}

\end{document}
