\documentclass[10pt,a4paper]{article}
\usepackage[utf8]{inputenc}

% Define the page margin
\usepackage[margin=3cm]{geometry}

% Better typography (font rendering)
\usepackage{microtype}

% Math environments and macros
\usepackage{amsmath}
\usepackage{amsfonts}
\usepackage{amssymb}
\usepackage{amsthm}

% Define \includegraphics to include graphics
\usepackage{graphicx}

% Draw graphics from a text description
\usepackage{tikz}

% Syntax highlighting
\usepackage{minted}

% Set global minted options
\setminted{linenos, autogobble, frame=lines, framesep=2mm}

% Import the comment environment for orgtbl-mode
\usepackage{comment}

% Do not indent paragraphs
\usepackage{parskip}

\title{Robotics, Sheet 2}
\author{Marten Lienen (03670270)}

\begin{document}

\maketitle

\section*{Exercise 1}

\subsection*{Part a)}

The rotation of the endeffector relative to the ground frame is just the sum of the rotations of all intermediate joints in the $x$-$y$ plane.
So $\Theta_{tip} = \Theta_{1} + \Theta_{2} + \Theta_{3}$.
The position is the sum of the rotated arm lengths.
\begin{align*}
  \begin{pmatrix}
    x\\y
  \end{pmatrix} = & \begin{pmatrix}
    \cos(\Theta_{1}) & -\sin(\Theta_{1})\\
    \sin(\Theta_{1}) & \cos(\Theta_{1})
  \end{pmatrix} \begin{pmatrix}
    l_{1}\\0
  \end{pmatrix} + \begin{pmatrix}
    \cos(\Theta_{1} + \Theta_{2}) & -\sin(\Theta_{1} + \Theta_{2})\\
    \sin(\Theta_{1} + \Theta_{2}) & \cos(\Theta_{1} + \Theta_{2})
  \end{pmatrix} \begin{pmatrix}
    l_{2}\\0
  \end{pmatrix} + \begin{pmatrix}
    \cos(\Theta_{1} + \Theta_{2} + \Theta_{3}) & -\sin(\Theta_{1} + \Theta_{2} + \Theta_{3})\\
    \sin(\Theta_{1} + \Theta_{2} + \Theta_{3}) & \cos(\Theta_{1} + \Theta_{2} + \Theta_{3})
  \end{pmatrix} \begin{pmatrix}
    l_{3}\\0
  \end{pmatrix}\\
  = & \begin{pmatrix}
    \cos(\Theta_{1})l_{1} + \cos(\Theta_{1} + \Theta_{2})l_{2} + \cos(\Theta_{1} + \Theta_{2} + \Theta_{3})l_{3}\\
    \sin(\Theta_{1})l_{1} + \sin(\Theta_{1} + \Theta_{2})l_{2} + \sin(\Theta_{1} + \Theta_{2} + \Theta_{3})l_{3}
  \end{pmatrix}
\end{align*}

\subsection*{Part b)}

\begin{equation*}
  J(\Theta) = \begin{pmatrix}
    -s_{1}l_{1} - s_{12}l_{2} - s_{123}l_{3} & -s_{12}l_{2} - s_{123}l_{3} & -s_{123}l_{3}\\
    c_{1}l_{1} + c_{12}l_{2} + c_{123}l_{3} & c_{12}l_{2} + c_{123}l_{3} & c_{123}l_{3}\\
    1 & 1 & 1
  \end{pmatrix}
\end{equation*}

\subsection*{Part c)}

Just apply the chain rule.
\begin{equation*}
  \dot{p}(\Theta, \dot{\Theta}) = J(\Theta) \cdot \dot{\Theta}
\end{equation*}

\subsection*{Part d)}

\begin{align*}
  \det(J(\Theta)) & = (-s_{1}l_{1} - s_{12}l_{2} - s_{123}l_{3})(c_{12}l_{2} + c_{123}l_{3}) + (-s_{12}l_{2} - s_{123}l_{3})(c_{123}l_{3}) + (-s_{123}l_{3})(c_{1}l_{1} + c_{12}l_{2} + c_{123}l_{3}) - (c_{12}l_{2} + c_{123}l_{3})(-s_{123}l_{3})\\
                  & \quad - (c_{123}l_{3})(-s_{1}l_{1} - s_{12}l_{2} - s_{123}l_{3}) - (c_{1}l_{1} + c_{12}l_{2} + c_{123}l_{3})(-s_{12}l_{2} - s_{123}l_{3})\\
                  & = 0 \Leftrightarrow \det({}^{3}J(\Theta)) = 0
\end{align*}
\begin{equation*}
  {}^{3}J(\Theta) = \begin{pmatrix}
    & & \\
    & & \\
    0 & 0 & 1
  \end{pmatrix}
\end{equation*}

\subsection*{Part e)}

\section*{Exercise 2}

\begin{align*}
  J^{T}J = & \begin{pmatrix}
    l_{1}s_{2} & l_{1}c_{2} + l_{2}\\
    0 & l_{2}
  \end{pmatrix} \begin{pmatrix}
    l_{1}s_{2} & 0\\
    l_{1}c_{2} + l_{2} & l_{2}
  \end{pmatrix}\\
  = & \begin{pmatrix}
    \left( l_{1}s_{2} \right)^{2} + \left( l_{1}c_{2} + l_{2} \right)^{2} & l_{1}l_{2}c_{2} + l_{2}^{2}\\
    l_{1}l_{2}c_{2} + l_{2}^{2} & l_{2}^{2}
  \end{pmatrix} = \delta I\\
  & \Leftrightarrow c_{2} = -\frac{l_{2}}{l_{1}} \land \left( l_{1}s_{2} \right)^{2} + \left( l_{1}c_{2} + l_{2} \right)^{2} = l_{2}^{2}
\end{align*}

Plugging in the first into the second equation gives
\begin{align*}
  \left( l_{1}s_{2} \right)^{2} + \left( l_{1}c_{2} + l_{2} \right)^{2} = & \left( l_{1}s_{2} \right)^{2} + \left( -l_{1}\frac{l_{2}}{l_{1}} + l_{2} \right)^{2}\\
  = & \left( l_{1}s_{2} \right)^{2} = l_{2}^{2} \Leftrightarrow s_{2} \in \{ -\frac{l_{2}}{l_{1}}, \frac{l_{2}}{l_{1}} \}
\end{align*}

Combining this intermediate result with the first equation gets us to
\begin{equation*}
  c_{2} = s_{2} \lor c_{2} = -s_{2} \Rightarrow \theta_{2} \in \{ \left( n - \frac{3}{4} \right)\pi \mid n \in \mathbb{Z} \}
\end{equation*}

The isotropic configurations are when the manipulator is either rotated $45^{\circ}$ rotation clock-wise or $135^{\circ}$ counter-clockwise, i.e. when it is orthogonal to the previous joint.
In these configurations the joints move the manipulator in orthogonal directions.

\section*{Exercise 3}

\subsection*{Differentiation}

The gripper position is given by
\begin{equation*}
  p(\Theta_{1}, \Theta_{2}) = \begin{pmatrix}
    \cos(\Theta_{1})l_{1} + \cos(\Theta_{1} + \Theta_{2})l_{2}\\
    \sin(\Theta_{1})l_{1} + \sin(\Theta_{1} + \Theta_{2})l_{2}
  \end{pmatrix}
\end{equation*}
Then we can compute the Jacobian through differentiation
\begin{equation*}
  J = \begin{pmatrix}
    -s_{1}l_{1} - s_{12}l_{2} & -s_{12}l_{2}\\
    c_{1}l_{1} + c_{12}l_{2} & c_{12}l_{2}
  \end{pmatrix}
\end{equation*}

\subsection*{Transformation}

First we have to determine the rotation matrix from frame $3$ to $0$.
It is simple in this scenario because the robot only moves in the $x$-$y$ plane.
\begin{equation*}
  {}^{0}_{3}R = \begin{pmatrix}
    \cos(\Theta_{1} + \Theta_{2}) & -\sin(\Theta_{1} + \Theta_{2})\\
    \sin(\Theta_{1} + \Theta_{2}) & \cos(\Theta_{1} + \Theta_{2})
  \end{pmatrix}
\end{equation*}
\begin{align*}
  {}^{0}_{3}R{}^{3}J(\Theta) = & \begin{pmatrix}
    c_{12} & -s_{12}\\
    s_{12} & c_{12}
  \end{pmatrix} \begin{pmatrix}
    s_{2}l_{1} & 0\\
    c_{2}l_{1} + l_{2} & l_{2}
  \end{pmatrix}\\
  = & \begin{pmatrix}
    c_{12}s_{2}l_{1} - s_{12}(c_{2}l_{1} + l_{2}) & -s_{12}l_{2}\\
    s_{12}s_{2}l_{1} + c_{12}(c_{2}l_{1} + l_{2}) & c_{12}l_{2}
  \end{pmatrix}\\
  = & \begin{pmatrix}
    c_{12}s_{2}l_{1} - s_{12}c_{2}l_{1} - s_{12}l_{2} & -s_{12}l_{2}\\
    s_{12}s_{2}l_{1} + c_{12}c_{2}l_{1} + c_{12}l_{2} & c_{12}l_{2}
  \end{pmatrix}\\
  = & \begin{pmatrix}
    (c_{12}s_{2} - s_{12}c_{2})l_{1} - s_{12}l_{2} & -s_{12}l_{2}\\
    (s_{12}s_{2} + c_{12}c_{2})l_{1} + c_{12}l_{2} & c_{12}l_{2}
  \end{pmatrix}\\
  = & \begin{pmatrix}
    ((c_{1}c_{2} - s_{1}s_{2})s_{2} - (s_{1}s_{2} + c_{1}c_{2})c_{2})l_{1} - s_{12}l_{2} & -s_{12}l_{2}\\
    ((s_{1}s_{2} + c_{1}c_{2})s_{2} + (c_{1}c_{2} - s_{1}s_{2})c_{2})l_{1} + c_{12}l_{2} & c_{12}l_{2}
  \end{pmatrix} = \dots = {}^{0}J(\Theta)
\end{align*}

\subsection*{Singularities}

\begin{align*}
  \det({}^{0}J) = -s_{1}c_{12}l_{1}l_{2} - s_{12}c_{12}l_{2}^{2} + s_{12}c_{1}l_{1}l_{2} + s_{12}c_{12}l_{2}^{2} = -s_{1}c_{12}l_{1}l_{2} + s_{12}c_{1}l_{1}l_{2} = (-s_{1}c_{12} + s_{12}c_{1})l_{1}l_{2}
\end{align*}
\begin{equation*}
  (-s_{1}c_{12} + s_{12}c_{1})l_{1}l_{2} = 0 \Leftrightarrow s_{12}c_{1} = s_{1}c_{12}
\end{equation*}

The Jacobian in frame $3$ has a simpler form and therefore a simpler determinant.
It also lets us compute the singularities way easier and we get $\Theta_{2} \in \{ 0, \frac{\pi}{2} \}$.

\end{document}
