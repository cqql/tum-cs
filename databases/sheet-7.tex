\documentclass[10pt,a4paper]{article}
\usepackage[utf8]{inputenc}

% Define the page margin
\usepackage[margin=3cm]{geometry}

% Better typography (font rendering)
\usepackage{microtype}

% Math environments and macros
\usepackage{amsmath}
\usepackage{amsfonts}
\usepackage{amssymb}
\usepackage{amsthm}

% Define \includegraphics to include graphics
\usepackage{graphicx}

% Draw graphics from a text description
\usepackage{tikz}

% Syntax highlighting
\usepackage{minted}

% Set global minted options
\setminted{linenos, autogobble, frame=lines, framesep=2mm}

% Import the comment environment for orgtbl-mode
\usepackage{comment}

% Do not indent paragraphs
\usepackage{parskip}

\title{Databases, Sheet 7}
\author{Marten Lienen (03670270)}

\begin{document}

\maketitle

\section*{Exercise 1}

Ein Mensch darf ein mal als Kind an Eltern\_von aber beliebig oft als Mutter oder Vater daran teilnehmen.
Genau genommen müsste jeder Mensch genau einmal daran teilnehmen, aber das würde zu einer unendlich großen Relation führen und lässt sich mit den genannten Mitteln auch nicht umsetzen.

An der verheiratet-Relation kann jeder Mensch ein oder kein mal teilnehmen pro Seite.
Wenn man es genau abbilden wollte, duerfte die Summe der Teilnahmen auf der linken oder rechten Seite nur 0 oder 1 sein, aber das lässt sich hier wieder nicht umsetzen.

\begin{minted}{sql}
  CREATE TABLE Menschen (
    SozVNR INT NOT NULL PRIMARY KEY,
    Name Text NOT NULL
  );

  CREATE TABLE Eltern_von (
    Kind INT NOT NULL,
    Mutter INT NOT NULL,
    Vater INT NOT NULL,
    UNIQUE(Kind),
    FOREIGN KEY (Kind) REFERENCES Menschen (SozVNR)
      ON UPDATE CASCADE ON DELETE CASCADE,
    FOREIGN KEY (Mutter) REFERENCES Menschen (SozVNR)
      ON UPDATE CASCADE ON DELETE CASCADE,
    FOREIGN KEY (Vater) REFERENCES Menschen (SozVNR)
      ON UPDATE CASCADE ON DELETE CASCADE
  );

  CREATE TABLE verheiratet (
    PartnerA INT NOT NULL,
    PartnerB INT NOT NULL,
    UNIQUE(PartnerA),
    UNIQUE(PartnerB),
    FOREIGN KEY (PartnerA) REFERENCES Menschen (SozVNR)
      ON UPDATE CASCADE ON DELETE CASCADE,
    FOREIGN KEY (PartnerB) REFERENCES Menschen (SozVNR)
      ON UPDATE CASCADE ON DELETE CASCADE
  );
\end{minted}

\section*{Exercise 2}

\subsection*{Part a)}

If this returns an empty row set, the pair $(A, B)$ has unique values in every row.
\begin{minted}{sql}
  SELECT A, B, COUNT(*) AS count
  FROM R
  GROUP BY A, B
  HAVING count > 1
\end{minted}

\subsection*{Part b)}

If this returns an empty row set, all rows with a specific pair of values $(d, e)$ in the columns $(D, E)$ contain the same value $B = b$ and thus $DE \rightarrow B$.
\begin{minted}{sql}
  SELECT D, E, MIN(B) AS min, MAX(B) AS max
  FROM R
  GROUP BY D, E
  HAVING min != max
\end{minted}

\section*{Exercise 3}

\subsection*{Part 1)}

\begin{itemize}
\item $\forall \alpha \subset PunkteListe : \forall \beta \subset \alpha : \alpha \rightarrow \beta$
\item $\{ Name \} \rightarrow \{ Klausursumme, KNote, Bonus, GNote \}$
\item Und viele mehr. Alleine die erste Regel generiert schon $2^{7}$ FDs nur für $\alpha = PunkteListe$.
\end{itemize}

\subsection*{Part 2)}

\begin{itemize}
\item $\{ Name, Aufgabe \}$
\item $\{ Name, Max, Erzielt \}$
\end{itemize}

\section*{Exercise 4}

\subsection*{Part a)}

\begin{equation*}
  H(F, A) = \{ A, B, C, D, E, F \}
\end{equation*}

\subsection*{Part b)}

\begin{itemize}
\item $\{ A \}$ offensichtlich
\item $\{ C \}$ weil $A$ funktional von $C$ abhängt und somit alles
\item $\{ E \}$ aus dem selben Grund wie $C$
\item $\{ F \}$ weil $C$ funktional von $F$ abhängt
\end{itemize}

$B$ und $D$ sind keine Kandidatenschlüssel, weil von ihnen nichts funktional abhängt.
Von $D$ hängt alles zusammen mit $C$ ab, aber $C$ alleine ist schon Kandidatenschlüssel, also kann $CD$ keiner sein, weil er nicht minimal ist.

\subsection*{Part c)}

\subsubsection*{Linksreduktion}

\begin{itemize}
\item $A \rightarrow BC$
\item $C \rightarrow AD$
\item $C \rightarrow BEF$
\item $E \rightarrow ABC$
\item $F \rightarrow CD$
\end{itemize}

\subsubsection*{Rechtsreduktion}

\begin{itemize}
\item $A \rightarrow BC$
\item $C \rightarrow \{  \}$
\item $C \rightarrow EF$
\item $E \rightarrow A$
\item $F \rightarrow CD$
\end{itemize}

\subsubsection*{Entfernen von leeren Regeln}

\begin{itemize}
\item $A \rightarrow BC$
\item $C \rightarrow EF$
\item $E \rightarrow A$
\item $F \rightarrow CD$
\end{itemize}

Vereinigungen sind nicht möglich/nötig.

\subsection*{Part d)}

Zuerst erzeugen wir Schemata für alle FDs in der kanonischen Überdeckung aus Teil c.
\begin{itemize}
\item $RA = \{ A, B, C \}$ mit
\item $RC = \{ C, E, F \}$
\item $RE = \{ E, A \}$
\item $RF = \{ F, C, D \}$
\end{itemize}

Da mehrere dieser Schemata Kandidatenschlüssel enthalten, sind wir fertig.

\end{document}
