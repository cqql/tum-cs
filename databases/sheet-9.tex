\documentclass[10pt,a4paper]{article}
\usepackage[utf8]{inputenc}

% Define the page margin
\usepackage[margin=3cm]{geometry}

% Better typography (font rendering)
\usepackage{microtype}

% Math environments and macros
\usepackage{amsmath}
\usepackage{amsfonts}
\usepackage{amssymb}
\usepackage{amsthm}

% Define \includegraphics to include graphics
\usepackage{graphicx}

% Draw graphics from a text description
\usepackage{tikz}

% Syntax highlighting
\usepackage{minted}

% Set global minted options
\setminted{linenos, autogobble, frame=lines, framesep=2mm}

% Import the comment environment for orgtbl-mode
\usepackage{comment}

% Do not indent paragraphs
\usepackage{parskip}

\title{Databases, Sheet 9}
\author{Marten Lienen (03670270)}

\begin{document}

\maketitle

\section*{Exercise 1}

\subsection*{1. Normalform}

\begin{equation*}
  R = (A, B, C) \qquad F = \{ A \rightarrow C \}
\end{equation*}

$AB$ ist Kandidatenschlüssel, aber $C$ ist nicht voll funktional abhängig von $AB$.

\subsection*{2. Normalform}

\begin{equation*}
  R = (A, B, C) \qquad F = \{ A \rightarrow ABC, B \rightarrow C \}
\end{equation*}

$C$ hängt funktional von $B$ ab, aber $C$ ist weder Teil eines Kandidatenschlüssels, noch Teil der Schlüssel, noch ist $B$ in Superschlüssel.
$R$ ist in zweiter Normalform, weil $A$ der einzige Kandidatenschlüssel ist und sowohl $B$ als auch $C$ funktional von $A$ abhängen.

\subsection*{3. Normalform}

\begin{equation*}
  R = (A, B, C) \qquad F = \{ B \rightarrow C \}
\end{equation*}

Es ist in dritter Normalform, weil $B$ in dem Kandidatenschlüssel $AB$ enthalten ist.
Es ist aber nicht in BCNF, weil $B$ gleichzeitig weder Superschlüssel ist, noch ist $B \rightarrow C$ trivial.

\subsection*{BC-Normalform}

\begin{equation*}
  R = (A, B, C) \qquad F = \{ A \rightarrow \rightarrow B \}
\end{equation*}

Das ist in dritter Normalform, weil es keine nichttrivialen funktionalen Abhängigkeiten gibt.
Es verletzt jedoch die BCNF, da $A$ nicht Superschlüssel und die MVD nicht trivial ist.

\subsection*{4. Normalform}

\begin{equation*}
  R = (A) \qquad F = \{  \}
\end{equation*}

Diese Relation ist in jeder Normalform.

\section*{Exercise 2}

\begin{equation*}
  R_{1} = (A, B, C) \qquad F_{1} = \{ AB \rightarrow C \}
\end{equation*}
\begin{equation*}
  R_{2} = (A, B, D) \qquad F_{2} = \{ AD \rightarrow B, AB \rightarrow D \}
\end{equation*}

\section*{Exercise 3}

\subsection*{Part a)}

\begin{minted}{sql}
  INSERT INTO kinder_fahrraeder (person, kind_name, kind_alter, fahrrad_typ, fahrrad_farbe)
  SELECT DISTINCT person, "Laura", 0, fahrrad_typ, fahrrad_farbe
  FROM kinder_fahrraeder
  WHERE person = "Thomas"
\end{minted}

\subsection*{Part b)}

\begin{equation*}
  |\{ c | [a, \_, c] \in R \}|
\end{equation*}
Also eines für jeden Wert von $c$, der mit $a$ verbunden ist.

\subsection*{Part c)}

Dann muss er auch seine Kinder verkaufen.

\subsection*{Part d)}

\begin{minted}{sql}
  CREATE TABLE kinder (
    person,
    kind_name,
    kind_alter
  );

  CREATE TABLE fahrraeder (
    person,
    fahrrad_typ,
    fahrrad_farbe
  );
\end{minted}

\subsection*{Part e)}

\begin{minted}{sql}
  SELECT *
  FROM kf
  WHERE EXISTS (
    SELECT *
    FROM kf2
    WHERE NOT EXISTS (
      SELECT *
      FROM kf3
      WHERE kf3.kind_name = kf.kind_name
        AND kf3.kind_alter = kf.kind_alter
        AND kf3.fahrrad_typ = kf2.fahrrad_typ
        AND kf3.fahrrad_farbe = kf2.fahrrad_farbe
        AND kf3.person = kf.person
    ) AND kf2.person = kf.person
  )
\end{minted}
Also wir wählen alle Kinder, für die es ein Fahrrad gibt, sodass keine Zeile existiert, die Kind und Fahrrad verknüpft.
Wenn das Abfrageergebnis leer ist, ist die MVD $person \rightarrow \rightarrow \{ kind-name, kind-alter \}$ erfüllt, weil es dann für jedes Kind und jedes Fahrrad eine Zeile gibt, die beide verbindet.

Leicht umgeschrieben für die andere MVD sieht die Abfrage folgendermaßen aus.
\begin{minted}{sql}
  SELECT *
  FROM kf
  WHERE EXISTS (
    SELECT *
    FROM kf2
    WHERE NOT EXISTS (
      SELECT *
      FROM kf3
      WHERE kf3.kind_name = kf2.kind_name
        AND kf3.kind_alter = kf2.kind_alter
        AND kf3.fahrrad_typ = kf.fahrrad_typ
        AND kf3.fahrrad_farbe = kf.fahrrad_farbe
        AND kf3.person = kf.person
    ) AND kf2.person = kf.person
  )
\end{minted}
Hier schauen wir nun, ob jedes Fahrrad mit jedem Kind verbunden ist.

\section*{Exercise 4}

\begin{equation*}
  F = \{ TicketID \rightarrow Festival, Festival \rightarrow \rightarrow TicketID, Festival \rightarrow \rightarrow Band, TicketID \rightarrow Festival \}
\end{equation*}

Man kann den Dekompositionsalgorithmus anwenden und die Relation in die vierte Normalform bringen.

\begin{equation*}
  R_{1} = (Festival, TicketID) \qquad F_{1} = \{ TicketID \rightarrow Festival \}
\end{equation*}
\begin{equation*}
  R_{2} = (Festival, Band) \qquad F_{2} = \{  \}
\end{equation*}

\section*{Exercise 5}

\subsection*{Part a)}

RAID5, weil es blockweise gestripet ist und die Paritätsinformationen verteilt sind.

\subsection*{Part b)}

Ein Verlust von einer Festplatte ist abgesichert, weil die Informationen aus den anderen Blöcken und den Paritätsinformationen wiederhergestellt werden können.
Ab zwei Ausfällen tritt Datenverlust auf, unabhängig von $n$.

\subsection*{Part c)}

Ja, z.B. indem man es mit einem RAID1 verbindet und ganze Blöcke spiegelt.

\subsection*{Part d)}

Man kann Blöcke einer Datei von mehreren Festplatten parallel lesen und dadurch erhöhte Leseperformanz erreichen.

\subsection*{Part e)}

% BEGIN RECEIVE ORGTBL exercise-5-e
\begin{tabular}{lr}
Block & Inhalt\\
\hline
C & 1000\\
P(DF) & 0101\\
H & 1110\\
\end{tabular}
% END RECEIVE ORGTBL exercise-5-e
\begin{comment}
#+ORGTBL: SEND exercise-5-e orgtbl-to-latex :splice nil :skip 0
| Block | Inhalt |
|-------+--------|
| C     |   1000 |
| P(DF) |   0101 |
| H     |   1110 |
\end{comment}


\end{document}
