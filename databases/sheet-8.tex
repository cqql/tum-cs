\documentclass[10pt,a4paper]{article}
\usepackage[utf8]{inputenc}

% Define the page margin
\usepackage[margin=3cm]{geometry}

% Better typography (font rendering)
\usepackage{microtype}

% Math environments and macros
\usepackage{amsmath}
\usepackage{amsfonts}
\usepackage{amssymb}
\usepackage{amsthm}

% Define \includegraphics to include graphics
\usepackage{graphicx}

% Draw graphics from a text description
\usepackage{tikz}

% Syntax highlighting
\usepackage{minted}

% Set global minted options
\setminted{linenos, autogobble, frame=lines, framesep=2mm}

% Import the comment environment for orgtbl-mode
\usepackage{comment}

% Do not indent paragraphs
\usepackage{parskip}

\title{Databases, Sheet 8}
\author{Marten Lienen (03670270)}

\begin{document}

\maketitle

\section*{Exercise 1}

Nein, $Fc$ ist nicht eindeutig.
Man betrachte eine Relation mit den Spalten $(a, b, c)$ und den funktionalen Abhängigkeiten $a \rightarrow (a, b)$ und $c \rightarrow (b, c)$.
Zwei verschiedene kanonische Überdeckungen dieser Abhängigkeiten sind $(a \rightarrow a, c \rightarrow (b, c))$ und $(a \rightarrow (a, b), c \rightarrow c)$.

\section*{Exercise 2}

\subsection*{FD für MaxSumme}

Da der Wert von MaxSumme in jeder Zeile gleich ist, ist MaxSumme funktional von jeder Spalte abhängig.
Modellmäßig sinnig wäre z.B. die FD $Name \rightarrow MaxSumme$.

\subsection*{Kanonische Überdeckung}

\subsubsection*{Linksreduktion}

Keine FD kann linksreduziert werden.

\subsubsection*{Rechtsreduktion}

Die einzigmögliche Rechtsreduktion ist
\begin{equation*}
  \{ Name \} \rightarrow \{ ErzieltSumme, Bonus \}
\end{equation*}
also das Streichen von $GNote$.

\subsubsection*{Streichen leerer FDs}

Es gibt keine.

\subsubsection*{Zusammenfassen}

Die einzigen FDs, die zusammengefasst werden können, sind $\{ Name \} \rightarrow \{ ErzieltSumme, Bonus \}$ und $\{ Name \} \rightarrow \{ MaxSumme \}$ zu $\{ Name \} \rightarrow \{ ErzieltSumme, MaxSumme, Bonus \}$.

\subsection*{Synthesealgorithmus}

% BEGIN RECEIVE ORGTBL exercise-2-schemata
\begin{tabular}{ll}
Schema & FDs\\
\hline
$\{ KNote, Bonus, GNote \}$ & $\{ KNote, Bonus \} \rightarrow \{ GNote \}$\\
$\{ Aufgabe, Max \}$ & $\{ Aufgabe \} \rightarrow \{ Max \}$\\
$\{ ErzieltSumme, KNote \}$ & $\{ ErzieltSumme \} \rightarrow \{ KNote \}$\\
$\{ Name, Aufgabe, Erzielt \}$ & $\{ Name, Aufgabe \} \rightarrow \{ Erzielt \}$\\
$\{ Name, ErzieltSumme, MaxSumme, Bonus \}$ & $\{ Name \} \rightarrow \{ ErzieltSumme, MaxSumme, Bonus \}\\
\end{tabular}
% END RECEIVE ORGTBL exercise-2-schemata
\begin{comment}
#+ORGTBL: SEND exercise-2-schemata orgtbl-to-latex :splice nil :skip 0 :raw t
| Schema                                      | FDs                                                         |
|---------------------------------------------+-------------------------------------------------------------|
| $\{ KNote, Bonus, GNote \}$                 | $\{ KNote, Bonus \} \rightarrow \{ GNote \}$                |
| $\{ Aufgabe, Max \}$                        | $\{ Aufgabe \} \rightarrow \{ Max \}$                       |
| $\{ ErzieltSumme, KNote \}$                 | $\{ ErzieltSumme \} \rightarrow \{ KNote \}$                |
| $\{ Name, Aufgabe, Erzielt \}$              | $\{ Name, Aufgabe \} \rightarrow \{ Erzielt \}$             |
| $\{ Name, ErzieltSumme, MaxSumme, Bonus \}$ | $\{ Name \} \rightarrow \{ ErzieltSumme, MaxSumme, Bonus \} |
\end{comment}

\section*{Exercise 3}

\subsection*{Kanonische Überdeckung}

\subsubsection*{Linksreduktion}

Grundsätzlich kann man nur FDs mit mehr als einer Variablen reduzieren.
Wir reduzieren $DE \rightarrow B$ zu $E \rightarrow B$.

\subsubsection*{Rechtsreduktion}

% BEGIN RECEIVE ORGTBL exercise-3-reductions
\begin{tabular}{ll}
FD & Reduziert\\
\hline
$DE \rightarrow B$ & $DE \rightarrow \emptyset$\\
\end{tabular}
% END RECEIVE ORGTBL exercise-3-reductions
\begin{comment}
#+ORGTBL: SEND exercise-3-reductions orgtbl-to-latex :splice nil :skip 0 :raw t
| FD                 | Reduziert                  |
|--------------------+----------------------------|
| $DE \rightarrow B$ | $DE \rightarrow \emptyset$ |
\end{comment}

\subsubsection*{Entfernen von leeren FDs}

Die FD wird dann auch direkt entfernt.

\subsubsection*{Zusammenfassen}

Man kann die FDs von $A$ zusammenfassen zu $A \rightarrow BCDE$.

\subsection*{Schemata}

% BEGIN RECEIVE ORGTBL exercise-3-schemata
\begin{tabular}{ll}
Schema & FDs\\
\hline
 & \\
\end{tabular}
% END RECEIVE ORGTBL exercise-3-schemata
\begin{comment}
#+ORGTBL: SEND exercise-3-schemata orgtbl-to-latex :splice nil :skip 0
| Schema  | FDs                  |
|---------+----------------------|
| $ABCDE$ | $A \rightarrow BCDE$ |
| $AF$    | $F \rightarrow A$    |
| $BEF$   | $E \rightarrow BF$   |
| $AC$    | $C \rightarrow A$    |
\end{comment}

Die letzte Relation kann gelöscht werden, weil sie vollständig in der ersten enthalten ist.

Man braucht auch keine weitere Relation, weil $AF$ ein Kandidatenschlüssel ist und es eine Relation gibt, die $AF$ enthält.

\end{document}
