\documentclass[10pt,a4paper]{article}
\usepackage[utf8]{inputenc}

% Define the page margin
\usepackage[margin=3cm]{geometry}

% Better typography (font rendering)
\usepackage{microtype}

% Math environments and macros
\usepackage{amsmath}
\usepackage{amsfonts}
\usepackage{amssymb}
\usepackage{amsthm}

% Define \includegraphics to include graphics
\usepackage{graphicx}

% Draw graphics from a text description
\usepackage{tikz}

% Syntax highlighting
\usepackage{minted}

% Set global minted options
\setminted{linenos, autogobble, frame=lines, framesep=2mm}

\title{Databases, Sheet 3}
\author{Marten Lienen (03670270)}

\begin{document}

\maketitle

\section*{Exercise 1}

\subsection*{Part a)}

\begin{itemize}
\item $A \times C \rightarrow B$
\item $A \times B \rightarrow C$
\end{itemize}

\subsection*{Part b)}

3 Tabellen für A, B und C mit jeweils der ID-Spalte und eine vierte mit jeweils allen drei IDs, um die Relation zu modellieren.

\subsection*{Part c)}

Schlüssel in R sind dabei $\{ \{ c_{id} \}, \{ b_{id} \}, \{ a_{id}, b_{id}, c_{id} \} \}$

\subsection*{Part d)}

Ich weiß nicht, was gemeint ist.
Meiner Ansicht nach, ist das ER-Modell vollständig umgesetzt: B und C können jeweils nur einmal an der Relation teilnehmen, A jedoch beliebig oft.
Dazu kann das gesamte Tripel nur einmal in der Relation auftauchen.

\section*{Exercise 2}

\subsection*{Part a)}

\begin{equation*}
  \pi_{Vorlesungen}(\sigma_{Student.Name = Xenokrates}(Studenten) \bowtie hoeren \bowtie Vorlesungen)
\end{equation*}

\subsection*{Part b)}

\begin{align*}
  & \pi_{V2.Titel}(\\
  & \quad \sigma_{V2.VorlNr = voraussetzen.Vorgaenger}(\\
  & \quad \quad \sigma_{V1.VorlNr = voraussetzen.Nachfolger}(\\
  & \quad \quad \quad \rho_{V1}(\sigma_{Vorlesungen.Titel = Wissenschaftstheorie}(Vorlesungen)) \times voraussetzen\\
  & \quad \quad ) \times \rho_{V2}(Vorlesungen)))
\end{align*}

\subsection*{Part c)}

\begin{align*}
  & \pi_{S1.Name, S2.Name}(\sigma_{S1.Name \ne S2.Name}(\\
  & \quad \sigma_{V.Titel = Grundzuege}(\rho_{S1}(Studenten) \bowtie hoeren \bowtie Vorlesungen)\\
  & \quad \times\\
  & \quad \sigma_{V.Titel = Grundzuege}(\rho_{S1}(Studenten) \bowtie hoeren \bowtie Vorlesungen)))
\end{align*}

\section*{Exercise 3}

\subsection*{Part a)}

\begin{equation*}
  Vorlesungen - \pi_{Vorlesungen}(Vorlesungen \bowtie hoeren)
\end{equation*}
Oder
\begin{equation*}
  \pi_{Vorlesungen}(\sigma_{hoeren = -}(Vorlesungen) left outer join hoeren)
\end{equation*}

\subsection*{Part b)}

\begin{align*}
  & Studenten - \pi_{Studenten}(\\
  & \qquad (Studenten \times Vorlesungen)\\
  & \qquad - \pi_{Studenten, Vorlesungen}(Studenten \bowtie hoeren \bowtie Vorlesungen))
\end{align*}
Wir berechnen zuerst alle Studenten mit allen Vorlesungen, die sie hoeren.
Indem wir dies von dem Kreuzprodukt aller Studenten und Vorlesungen abziehen, erhalten wir alle Studenten mit den Vorlesungen, die sie nicht hoeren.
Wenn wir diese nun wieder von allen Studenten abziehen, bleiben diejenigen, die alle Vorlesungen hoeren.

\section*{Exercise 4}

Die Darstellung ternärer durch binäre Beziehungen lässt die Datenbank weniger Integritätsbedingungen durchsetzen.

Wenn man ein Attribut als Entity mit Beziehung darstellt, kann man mehrere Attribute mit einer Entität verknüpfen.

\end{document}
