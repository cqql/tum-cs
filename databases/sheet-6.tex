\documentclass[10pt,a4paper]{article}
\usepackage[utf8]{inputenc}

% Define the page margin
\usepackage[margin=3cm]{geometry}

% Better typography (font rendering)
\usepackage{microtype}

% Math environments and macros
\usepackage{amsmath}
\usepackage{amsfonts}
\usepackage{amssymb}
\usepackage{amsthm}

% Define \includegraphics to include graphics
\usepackage{graphicx}

% Draw graphics from a text description
\usepackage{tikz}

% Syntax highlighting
\usepackage{minted}

% Set global minted options
\setminted{linenos, autogobble, frame=lines, framesep=2mm}

% Import the comment environment for orgtbl-mode
\usepackage{comment}

% Do not indent paragraphs
\usepackage{parskip}

\title{Databases, Sheet 6}
\author{Marten Lienen (03670270)}

\begin{document}

\maketitle

\section*{Exercise 1}

\begin{minted}{sql}
  WITH bekanntheit (m1, m2, g)
  AS (
    SELECT h1.MatrNr, h2.MatrNr, 1
    FROM hoeren h1
    INNER JOIN hoeren h2 ON h1.VorlNr = h2.VorlNr
    WHERE h1.MatrNr != h2.MatrNr
    UNION ALL
    SELECT b.m1, h2.MatrNr, 1 + b.g
    FROM bekanntheit b
    INNER JOIN hoeren h1 ON h1.MatrNr = b.m2
    INNER JOIN hoeren h2 ON h2.VorlNr = h1.VorlNr
    WHERE h2.MatrNr != b.m1
      AND b.g < (SELECT COUNT(*) FROM Studenten)
  )
  SELECT m1, m2, MIN(g) AS grad
  FROM bekanntheit
  GROUP BY m1, m2
  ORDER BY m1, grad DESC, m2
\end{minted}

\section*{Exercise 2}

\begin{minted}{sql}
  WITH sws (m, sum)
  AS (
    SELECT s.MatrNr, IFNULL(SUM(v.SWS), 0)
    FROM Studenten s
    LEFT JOIN hoeren h ON h.MatrNr = s.MatrNr
    LEFT JOIN Vorlesungen v ON v.VorlNr = h.VorlNr
    GROUP BY s.MatrNr
  )
  SELECT s.*, sws.sum
  FROM sws
  INNER JOIN Studenten s ON s.MatrNr = sws.m
  WHERE sws.sum > (SELECT AVG(sum) FROM sws)
\end{minted}

\section*{Exercise 3}

\subsection*{Part a)}

\begin{minted}{sql}
  WITH maenner (fak, sum)
  AS (
    SELECT FakName, SUM(*)
    FROM SudentenGF
    WHERE Geschlecht = "M"
    GROUP BY FakName
  )
  SELECT s.FakName, IFNULL(m.sum, 0) / COUNT(*)
  FROM StudentenGF s
  LEFT JOIN manner m ON m.fak = s.FakName
  GROUP BY s.FakName
\end{minted}

\subsection*{Part b)}

\begin{minted}{sql}
  WITH anzahl_vorlesungen (FakName, anzahl)
  AS (
    SELECT p.FakName, COUNT(*)
    FROM ProfessorenF p
    INNER JOIN Vorlesungen v ON v.gelesenVon = p.PersNr
    GROUP BY p.FakName
  )
  SELECT s.*, COUNT(*) anzahl
  FROM StudentenGF s
  INNER JOIN hoeren h ON h.MatrNr = s.MatrNr
  INNER JOIN Vorlesungen v ON v.VorlNr = h.VorlNr
  INNER JOIN ProfessorenF p ON p.PersNr = v.gelesenVon
  LEFT JOIN anzahl_vorlesungen av ON av.FakName = s.FakName
  WHERE p.FakName = s.FakName
  GROUP BY (s.MatrNr)
  HAVING anzahl = av.anzahl
\end{minted}

\begin{minted}{sql}
  WITH nichthoerer (MatrNr) AS (
    SELECT s.MatrNr
    FROM StudentenGF s
    JOIN Vorlesungen v
    INNER JOIN ProfessorenF p ON p.PersNr = v.gelesenVon
      AND p.FakName = s.FakName
    WHERE NOT EXISTS (SELECT h.* FROM hoeren h WHERE h.MatrNr = s.MatrNr AND h.VorlNr = v.VorlNr)
  )
  SELECT s.*
  FROM Studenten s
  LEFT JOIN nichthoerer nh ON nh.MatrNr = s.MatrNr
  WHERE nh.MatrNr IS NULL
\end{minted}

\section*{Exercise 4}

\begin{minted}{sql}
  WITH fruehstens (VorlNr, Titel, Semester) AS (
    SELECT v.VorlNr, v.Titel, 1
    FROM Vorlesungen v
    LEFT JOIN voraussetzen va ON va.Nachfolger = v.VorlNr
    WHERE va.Vorgaenger IS NULL
    UNION ALL
    SELECT v.VorlNr, v.Titel, f.Semester + 1
    FROM fruehstens f
    INNER JOIN voraussetzen va ON va.Vorgaenger = f.VorlNr
    INNER JOIN Vorlesungen v ON v.VorlNr = va.Nachfolger
  )
  SELECT DISTINCT *
  FROM fruehstens
\end{minted}

\end{document}
