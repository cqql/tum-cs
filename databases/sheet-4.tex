\documentclass[10pt,a4paper]{article}
\usepackage[utf8]{inputenc}

% Define the page margin
\usepackage[margin=3cm]{geometry}

% Better typography (font rendering)
\usepackage{microtype}

% Math environments and macros
\usepackage{amsmath}
\usepackage{amsfonts}
\usepackage{amssymb}
\usepackage{amsthm}

% Define \includegraphics to include graphics
\usepackage{graphicx}

% Draw graphics from a text description
\usepackage{tikz}

% Syntax highlighting
\usepackage{minted}

% Set global minted options
\setminted{linenos, autogobble, frame=lines, framesep=2mm}

% Import the comment environment for orgtbl-mode
\usepackage{comment}

\title{Databases, Sheet 4}
\author{Marten Lienen (03670270)}

\begin{document}

\maketitle

\section*{Exercise 1}

\subsection*{Part a)}

\begin{align*}
  \{ v | & v \in Vorlesungen\ \land\\
         & \forall s \in Studenten (\\
         & \quad s.Name = ``Xenokrates''\\
         & \quad \Rightarrow \exists h \in hoeren (h.MatrNr = s.MatrNr\ \land\ s.VorlNr = v.VorlNr)) \}
\end{align*}

\begin{align*}
  \{ v, t, s, g \mid & [v, t, s, g] \in Vorlesungen\ \land\\
                     & \forall [m, n, \_] \in Studenten (\\
                     & n = ``Xenokrates'' \Rightarrow \exists [m, v] \in hoeren) \}
\end{align*}

\subsection*{Part b)}

\begin{align*}
  \{ v.Titel \mid & v \in Vorlesungen\ \land\\
                  & \exists vr \in vorraussetzen (\\
                  & \quad vr.Vorgaenger = v.VorlNr\ \land\\
                  & \quad \exists v2 \in Vorlesungen (v2.VorlNr = vr.Nachfolger\ \land\ v2.Name = ``Wissenschaftstheorie'')) \}
\end{align*}

\begin{align*}
  \{ t \mid & [v, t, \_, \_], [v2, ``Wissenschaftstheorie'', \_, \_] \in Vorlesungen\ \land\\
            & [v, v2] \in voraussetzen \}
\end{align*}

\subsection*{Part c)}

\begin{align*}
  \{ s1.Name, s2.Name \mid & s1, s2 \in Studenten\ \land\\
                           & \forall v \in Vorlesungen (\\
                           & \quad v.Name = ``Grundzuege''\\
                           & \quad \rightarrow \exists h1, h2 \in hoeren (\\
                           & \quad \quad h1.MatrNr = s1.MatrNr\ \land\ h2.MatrNr = s2.MatrNr\\
                           & \quad \quad \land h1.VorlNr = h2.VorlNr = v.VorlNr)) \}
\end{align*}

\begin{align*}
  \{ [n1, n2] \mid & [m1, n1, \_], [m2, n2, \_] \in Studenten\ \land\\
                   & [v, ``Grundzuege'', \_, \_] \in Vorlesungen\ \land\\
                   & [m1, v], [m2, v] \in hoeren \}
\end{align*}

\section*{Exercise 2}

\subsection*{Relationenalgebra}

\begin{align*}
  & \Pi_{Studenten}(\\
  & \quad (Studenten \bowtie hoeren)\\
  & \quad \div\\
  & \quad \Pi_{Vorlesungen.VorlNr}(\\
  & \quad \quad \sigma_{Professoren.Name = ``Sokrates''}(\\
  & \quad \quad \quad Vorlesungen \bowtie_{gelesenVon = PersNr} Professoren)))
\end{align*}

\subsection*{Relationaler Tupelkalkül}

\begin{align*}
  \{ s | & s \in Studenten\ \land\\
         & \forall v \in Vorlesungen\ (\\
         & \quad \exists p \in Professoren (p.Name = ``Sokrates''\ \land\ v.gelesenVon = p.PersNr)\\
         & \quad \Rightarrow \exists h \in hoeren (h.MatrNr = s.MatrNr\ \land\ h.VorlNr = v.VorlNr)) \}
\end{align*}

\subsection*{Relationaler Domänenkalkül}

\begin{align*}
  \{ m, n, s | & \exists [m, n, s] \in Studenten\ \land\\
               & \forall [v, \_, \_, g] \in Vorlesungen\ (\\
               & \quad \exists ([g, name, \_, \_] \in Professoren\ \land\ name = ``Sokrates'')\\
               & \quad \Rightarrow \exists [m, v] \in hoeren) \}
\end{align*}

\section*{Exercise 3}

\subsection*{Part a)}

\begin{minted}{sql}
  SELECT s.*
  FROM Studenten s
  INNER JOIN hoeren h ON h.MatrNr = S.MatrNr
  INNER JOIN Vorlesungen v ON v.VorlNr = h.VorlNr
  INNER JOIN Professoren p ON p.PersNr = v.gelesenVon
  WHERE p.Name = "Sokrates"
\end{minted}

\subsection*{Part b)}

\begin{minted}{sql}
  SELECT s.*
  FROM Studenten s
  INNER JOIN hoeren h ON h.MatrNr = S.MatrNr
  WHERE h.VorlNr IN (
    SELECT v.VorlNr
    FROM Vorlesungen v
    INNER JOIN hoeren h2 ON h2.VorlNr = v.VorlNr
    INNER JOIN Studenten s2 ON s2.MatrNr = h2.MatrNr
    WHERE s2.Name = "Fichte"
  )
\end{minted}

\subsection*{Part c)}

\begin{minted}{sql}
  SELECT a.*
  FROM Assistenten a
  INNER JOIN Professoren p ON p.PersNr = a.Boss
  INNER JOIN Vorlesungen v ON v.gelesenVon = p.PersNr
  INNER JOIN hoeren h ON h.VorlNr = v.VorlNr
  INNER JOIN Studenten s ON s.MatrNr = h.MatrNr
  WHERE s.Name = "Fichte"
\end{minted}

\subsection*{Part d)}

\begin{minted}{sql}
  SELECT p.Name
  FROM Professoren p
  INNER JOIN Vorlesungen v ON v.gelesenVon = p.PersNr
  INNER JOIN hoeren h ON h.VorlNr = v.VorlNr
  INNER JOIN Studenten s ON s.MatrNr = h.MatrNr
  WHERE s.Name = "Xenokrates"
\end{minted}

\subsection*{Part e)}

\begin{minted}{sql}
  SELECT v.Titel
  FROM Vorlesungen v
  INNER JOIN hoeren h ON h.VorlNr = v.VorlNr
  INNER JOIN Studenten s ON s.MatrNr = h.MatrNr
  WHERE s.Semester >= 1 AND s.Semester <= 4
\end{minted}

\end{document}
