\documentclass[10pt,a4paper]{article}
\usepackage[utf8]{inputenc}

% Math environments and macros
\usepackage{amsmath}
\usepackage{amsfonts}
\usepackage{amssymb}
\usepackage{amsthm}

% Define the page margin
\usepackage[margin=3cm]{geometry}

% Define \includegraphics to include graphics
\usepackage{graphicx}

% Better typography (font rendering)
\usepackage{microtype}

% Syntax highlighting
\usepackage{minted}

% Set global minted options
\setminted{linenos, autogobble, frame=lines, framesep=2mm}

\title{Grundlagen: Datenbanken, Blatt 2}
\author{Marten Lienen (03670270)}

\begin{document}

\maketitle

\section*{Exercise 1}

\subsection*{Part a)}

\begin{align*}
  & \sigma_{Assistenten.Name}(\\
  & \qquad (((\sigma_{Name=Fichte}(Studenten) \bowtie hoeren)\\
  & \qquad \qquad \bowtie Vorlesungen) \bowtie_{gelesenVon=PersNr} Professoren) \bowtie_{Boss=PersNr} Assistenten)
\end{align*}

\subsection*{Part b)}

\begin{align*}
  & \sigma_{Studenten}(\sigma_{Studenten,Vorgaenger}(Studenten \bowtie hoeren \bowtie voraussetzen)\\
  & \qquad - \rho_{VorlNr \rightarrow Vorgaenger}(\sigma_{Studenten,VorlNr}(Studenten \bowtie hoeren)))
\end{align*}

\section*{Exercise 2}

\subsection*{Part a)}

\subsection*{Part b)}

\subsection*{Part c)}

\section*{Exercise 3}

\begin{enumerate}
\item 3
\item $T \times S \rightarrow U$, $U \times S \rightarrow T$ und $U \times T \rightarrow S$
\item
  1) Ein Student kann bei einem Tutor nur eine Übung haben und ein Tutor kann einem Studenten nur eine Übung geben
  2) Ein Student kann eine Übung nur bei einem Tutor haben und eine Übung kann einem Studenten nur von einem Tutor gegeben werden
  3) Ein Tutor kann eine Übung nur einem Studenten geben
\end{enumerate}

Hier gelten die partiellen Funktionen $S \times T \rightarrow U$ und $S \times U \rightarrow T$.

\section*{Exercise 4}

A -> N, C -> M, B -> 1.

Die Entitäten, die als Zielmenge vorkommen, kriegen eine 1.
Der Rest kriegt ein N.

Alle Entitäten, die eine 1 als Funktionalitätsangabe haben, können als partielle Funktion aus allen anderen Entitäten gefunden werden.
Sie implizieren also eine partielle Funktion von allen anderen Entitäten auf die eine.

\end{document}
