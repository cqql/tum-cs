\documentclass[10pt,a4paper]{article}
\usepackage[utf8]{inputenc}

% Define the page margin
\usepackage[margin=3cm]{geometry}

% Better typography (font rendering)
\usepackage{microtype}

% Math environments and macros
\usepackage{amsmath}
\usepackage{amsfonts}
\usepackage{amssymb}
\usepackage{amsthm}

% Define \includegraphics to include graphics
\usepackage{graphicx}

% Draw graphics from a text description
\usepackage{tikz}

% Syntax highlighting
\usepackage{minted}

% Set global minted options
\setminted{linenos, autogobble, frame=lines, framesep=2mm}

% Import the comment environment for orgtbl-mode
\usepackage{comment}

% Do not indent paragraphs
\usepackage{parskip}

\title{Automata and Formal Languages, Sheet 2}
\author{Marten Lienen (03670270)}

\begin{document}

\maketitle

\section*{Exercise 2.1}

\subsection*{Part a)}

\begin{proof}
  Let $L \subseteq \Sigma_{1}^{*}$ be regular.
  So there is a DFA $A = (Q, \Sigma_{1}, \delta_{B}, q_{0}, F)$ that accepts $L$.
  Then construct a new DFA $B = (Q, \Sigma_{2}, \delta_{B}, q_{0}, F)$ such that
  $\delta_{B}(q, )$
  \begin{equation*}
    \delta_{B} =
  \end{equation*}
  It remains to show that $B$ accepts $h(L)$.

  Let $w \in \Sigma_{1}^{*}$ and $w' = h(w) \in \Sigma_{2}^{*}$.
  Then $w = w_{1} \dots w_{n}$ and $$.
\end{proof}

\subsection*{Part b)}

\subsection*{Part c)}

\subsection*{Part d)}

\section*{Exercise 2.2}

\subsection*{$(ab + ba)^{*}$}

\begin{align*}
  L^{\epsilon} & = (ab + ba)^{*}\\
  L^{a} & = b(ab + ba)^{*}\\
  L^{b} & = a(ab + ba)^{*}\\
  L^{aa} & = \emptyset
\end{align*}

\subsection*{$(aa)^{*}$}

\begin{align*}
  L^{\epsilon} & = (aa)^{*}\\
  L^{a} & = a(aa)^{*}\\
  L^{b} & = \emptyset
\end{align*}

\subsection*{$\{ a^{n}b^{n}c^{n} \mid n \ge 0 \}$}

\begin{align*}
  L^{a^{k}} & = \{ a^{n - k}b^{n + k}c^{n + k} \mid n \ge 0 \} \forall k \ge 0\\
  L^{a^{k}b^{j}} & = b^{k - j}c^{k} \forall k \ge 1, j < k\\
  L^{a^{k}b^{k}c^{j}} & = c^{k - j} \forall k \ge 1, j < k\\
  L^{abb} & = \emptyset
\end{align*}

\section*{Exercise 2.3}

\subsection*{Part a)}

\subsubsection*{$(ab + ba)^{*}$}

\begin{align*}
  {}^{bb}L & = \emptyset\\
  {}^{a}L & = (ab + ba)^{*}b\\
  {}^{b}L & = (ab + ba)^{*}a\\
  {}^{\epsilon}L & = (ab + ba)^{*}
\end{align*}

\subsubsection*{$(aa)^{*}$}

\begin{align*}
  {}^{\epsilon}L & = (aa)^{*}\\
  {}^{a}L & = (aa)^{*}a\\
  {}^{b}L & = \emptyset
\end{align*}

\subsection*{Part b)}

\begin{proof}

\end{proof}

\subsection*{Part c)}

\section*{Exercise 2.4}

\subsection*{Part a)}

Let $L = U(S)$ for some $S \subseteq \mathbb{N}$ and $T = L^{U(n)}$ for some $n \in \mathbb{N}$.
\begin{equation*}
  T = \{ U(m) \mid m + n \in S \}
\end{equation*}

Let $L = B(S)$ for some $S \subseteq \mathbb{N}$ and $T = L^{U(n)}$ for some $n \in \mathbb{N}$.
\begin{equation*}
  T = \{ B(m) \mid B^{-1}(B(n) B(m)) \in S \} = \{ B(m) \mid n \cdot 2^{\lceil\log_{2}(m)\rceil} + m \in S \}
\end{equation*}

\subsection*{Part b)}

\begin{proof}
  Let $S = \{ 2^{n} \mid n \in \mathbb{N}_{0} \}$.
  Then the binary respectively unary encodings of $S$ are $B(S) = \{ 10^{n} \mid n \in \mathbb{N}_{0} \}$ respectively $U(S) = \{ \bullet^{2^{n}} \mid n \in \mathbb{N}_{0} \}$.
  Obviously $B(S)$ only has 3 residuals for $\epsilon$, $1$ and $0$.
  $U(S)$ on the other hand has infinitely many residuals:
  Let $P = U(S)^{\bullet^{2^{p}}}$ and $Q = U(S)^{\bullet^{2^{q}}}$ be two residuals of $U(S)$ for any $p, q \in \mathbb{N}_{0}$ with $p < q$.
  Then $\bullet^{2^{p}} \in P$ because $2^{p} + 2^{p} = 2^{p + 1}$, but $\bullet^{2^{p}} \not\in Q$ because $2^{p} + 2^{q} = 2^{p}(1 + 2^{q - p})$ which is not a power of two because $(1 + 2^{q - p})$ is uneven since $q > p \Leftrightarrow q - p > 0$.
  Therefore $U(S)$ is not regular but $B(S)$ is.
\end{proof}

\subsection*{Part c)}

\begin{proof}
  Let $S \subseteq \mathbb{N}$ such that $U(S)$ is regular.
  Hence there are only finitely many residuals with prefixes $u_{1}, \dots, u_{n}$.
  That means that for every $s \in S$ there is an $s + u_{j} \in S$ for some $j \in \{ 1, \dots, n \}$.
\end{proof}

\section*{Exercise 2.5}

\section*{Exercise 2.6}

\subsection*{Part a)}

\subsection*{Part b)}

\subsection*{Part c)}

\subsection*{Part d)}

\subsection*{Part e)}

\subsection*{Part f)}

\subsection*{Part g)}

\end{document}
