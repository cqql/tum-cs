%%% -*- TeX-master: "case-study.tex" -*-
\section{Conclusion and Outlook}
\label{sec:conclusion}

We have presented the generalized moment limiter for discontinuous Galerkin methods together with its foundation, the minmod limiter, and looked into numerical results, that showcase the limiters utility and effectiveness.

The limiter has further potential for theoretical analysis as well as transfer to related methods.
One should work out the details of the connection between the coefficients and the derivatives of the Legendre polynomials, since such an analysis could yield more insight into the choice of limiter parameters.
Alternatively one could try to use a different set of basis functions.
There could be a candidate for which the aforementioned connection can be more easily determined.
Related to the search for promising basis functions is the idea of building a high-order limiter for nodal DG methods.
