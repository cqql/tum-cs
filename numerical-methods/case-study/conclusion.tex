%%% -*- TeX-master: "case-study.tex" -*-
\section{Conclusion and Outlook}
\label{sec:conclusion}

\red{In this paper we have restated the relevant parts of discontinuous Galerkin methods to study limiters for controlling oscillations.
First, we introduced the minmod limiter, a simple approach for first-order methods with the TVD property.
Nonetheless, it is not able to deal satisfactorily with discontinuities.
In the end we have presented the moment limiter that extends the minmod limiter to higher-order methods.
A comparison by numerical experiment showed that the moment limiter delivers a significant improvement over its predecessor.
Additionally, we have looked into Total Variation analysis of limiters and how this justifies and even necessitates their use.}

The \red{moment} limiter has further potential for theoretical analysis as well as transfer to related methods.
One should work out the details of the connection between the coefficients and the derivatives of the Legendre polynomials, since such an analysis could yield more insight into the choice of limiter parameters.
Additionally one could research if and how the idea of successive limiting can be transferred to other basis functions.
