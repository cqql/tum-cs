\documentclass[10pt,a4paper]{article}
\usepackage[utf8]{inputenc}

% Define the page margin
\usepackage[margin=3cm]{geometry}

% Better typography (font rendering)
\usepackage{microtype}

% Math environments and macros
\usepackage{amsmath}
\usepackage{amsfonts}
\usepackage{amssymb}
\usepackage{amsthm}

% Define \includegraphics to include graphics
\usepackage{graphicx}

% Do not indent paragraphs
\usepackage{parskip}

\newtheorem{lemma}{Lemma}
\newtheorem{definition}{Definition}

\DeclareMathOperator{\mean}{mean}

\title{Semidefinite Programming Approaches to Clustering}
\author{Marten Lienen (03670270)}
\date{}

\begin{document}

\maketitle

\begin{definition}
  Let $P \in \mathbb{R}^{m \times N}$ denote the data matrix that has the data points as its columns.
\end{definition}

\begin{definition}
  Let $X_{D}$ be the minimizer of the SDP.
\end{definition}

\begin{definition}
  We say that $x_{i}$ and $x_{j}$ \emph{belong together} if ... $X_{D,ij} = \max_{k} X_{D,ik}$?
  This means that $x_{i}$ and $x_{j}$ probably belong to the same cluster according to the denoising procedure.
\end{definition}

\begin{definition}
  $\hat{\gamma}_{i} = \mean \{ c_{j} \mid x_{j}~\text{belongs to}~x_{i} \}$ where $c_{j}$ is the $j$-th column of $PX_{D}$.
  These should be the best cluster center approximators according to the estimated cluster affiliations.
\end{definition}

\begin{lemma}
  Construct an estimator $d_{\min} = \min \left\{ ||\hat{\gamma}_{a} - \hat{\gamma}_{b}|| \right\}$ for the minimum cluster center separation.
  Then $d_{\min} \le (1 + \varepsilon) \Delta_{\min}$ with probability $P(\varepsilon, ..)$.
\end{lemma}

Maybe find a statement for the $\tilde{\gamma}_{i}$s first.
$\tilde{\gamma}_{i}$ is $\hat{\gamma}_{i}$ if we knew the real cluster assignments.

If this lemma holds, we can use this $d_{\min}$ to choose an $\varepsilon < \frac{\Delta_{\min}}{8}$ for the rounding step.

\end{document}
