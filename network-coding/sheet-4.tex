\documentclass[10pt,a4paper]{article}
\usepackage[utf8]{inputenc}

% Define the page margin
\usepackage[margin=3cm]{geometry}

% Better typography (font rendering)
\usepackage{microtype}

% Math environments and macros
\usepackage{amsmath}
\usepackage{amsfonts}
\usepackage{amssymb}
\usepackage{amsthm}

% Define \includegraphics to include graphics
\usepackage{graphicx}

% Draw graphics from a text description
\usepackage{tikz}

% Syntax highlighting
\usepackage{minted}

% Set global minted options
\setminted{linenos, autogobble, frame=lines, framesep=2mm}

% Import the comment environment for orgtbl-mode
\usepackage{comment}

\title{Network Coding, Sheet 4}
\author{Marten Lienen (03670270)}

\begin{document}

\maketitle

\section*{Exercise 1}

\subsection*{Part a)}

Yes, because we have received three packets and the prepended coding coefficients are linearly independent.

\subsection*{Part b)}

\begin{equation*}
  \begin{pmatrix}
    02 & 12 & 22 & 01 & 01 & 01\\
    11 & 00 & 22 & 12 & 12 & 12\\
    10 & 11 & 10 & 10 & 11 & 12
  \end{pmatrix}
\end{equation*}
Multiply first row with $22$
\begin{equation*}
  \begin{pmatrix}
    11 & 02 & 20 & 22 & 22 & 22
  \end{pmatrix}
\end{equation*}
And with $20$
\begin{equation*}
  \begin{pmatrix}
    10 & 11 & 12 & 20 & 20 & 20
  \end{pmatrix}
\end{equation*}
\begin{equation*}
  \begin{pmatrix}
    02 & 12 & 22 & 01 & 01 & 01\\
    00 & 01 & 02 & 20 & 20 & 20\\
    00 & 00 & 01 & 20 & 21 & 22
  \end{pmatrix}
\end{equation*}
Divide first line by $02$
\begin{equation*}
  \begin{pmatrix}
    01 & 21 & 11 & 02 & 02 & 02\\
    00 & 01 & 02 & 20 & 20 & 20\\
    00 & 00 & 01 & 20 & 21 & 22
  \end{pmatrix}
\end{equation*}
Multiply the second row by $21$ and subtract it from the first one
\begin{equation*}
  \begin{pmatrix}
    00 & 21 & 12 & 22 & 22 & 22
  \end{pmatrix}
\end{equation*}
\begin{equation*}
  \begin{pmatrix}
    01 & 00 & 02 & 10 & 10 & 10\\
    00 & 01 & 02 & 20 & 20 & 20\\
    00 & 00 & 01 & 20 & 21 & 22
  \end{pmatrix}
\end{equation*}
Multiply the last row by $02$
\begin{equation*}
  \begin{pmatrix}
    00 & 00 & 02 & 10 & 12 & 11
  \end{pmatrix}
\end{equation*}
And subtract it from the first two rows
\begin{equation*}
  \begin{pmatrix}
    01 & 00 & 00 & 00 & 01 & 02\\
    00 & 01 & 00 & 10 & 11 & 12\\
    00 & 00 & 01 & 20 & 21 & 22
  \end{pmatrix}
\end{equation*}

\section*{Exercise 2}

\subsection*{Part a)}

The probability that we draw two linearly dependent vectors is
\begin{equation*}
  \frac{15}{16} \cdot \frac{2}{16} + \frac{1}{16} = \frac{30}{256} + \frac{16}{256} = \frac{46}{256}
\end{equation*}
Therefore the probability of drawing linearly independent vectors is
\begin{equation*}
  1 - \frac{46}{256} = \frac{210}{256} = \frac{105}{128} = \frac{15}{16} \cdot \frac{14}{16}
\end{equation*}

\subsection*{Part b)}

\begin{equation*}
  \frac{15}{16} \cdot \frac{14}{16} \cdot \frac{12}{16} \cdot \frac{8}{16} = \frac{315}{1024} \approx 0.3076
\end{equation*}

\subsection*{Part c)}

\begin{equation*}
  \prod_{i = 1}^{N} \frac{q^{N} - 1 - \sum_{k = 1}^{i - 1} \binom{i - 1}{k} (q - 1)^{k}}{q^{N}} = \prod_{i = 1}^{N} \frac{q^{N} - 1 - q^{i - 1} + 1}{q^{N}} = \prod_{i = 1}^{N} \frac{q^{N} - q^{i - 1}}{q^{N}}
\end{equation*}

\subsection*{Part d)}

\begin{align*}
  \prod_{i = 1}^{N} \frac{q^{N} - q^{i - 1}}{q^{N}} & = \prod_{i = 1}^{N} \left( 1 - q^{i - 1 - N} \right)\\
                                                    & = \prod_{i = 1}^{N} \left( 1 - \frac{1}{q^{N + 1 - i}} \right)\\
                                                    & = \prod_{i = 1}^{N} \left( 1 - \frac{1}{q^{i}} \right)\\
                                                    & = \prod_{i = 1}^{N} \left( 1 - \left( \frac{1}{q} \right)^{i} \right) \ge \Phi\left( \frac{1}{q} \right)
\end{align*}

\subsection*{Part e)}

% BEGIN RECEIVE ORGTBL evaluations
\begin{tabular}{rr}
q & $\Phi\left(\frac{1}{q}\right)$\\
\hline
2 & 0.289\\
4 & 0.689\\
16 & 0.934\\
256 & 0.996\\
\end{tabular}
% END RECEIVE ORGTBL evaluations
\begin{comment}
#+ORGTBL: SEND evaluations orgtbl-to-latex :splice nil :skip 0
|   q | $\Phi\left(\frac{1}{q}\right)$ |
|-----+--------------------------------|
|   2 |                          0.289 |
|   4 |                          0.689 |
|  16 |                          0.934 |
| 256 |                          0.996 |
\end{comment}

\end{document}
