\documentclass[10pt,a4paper]{article}
\usepackage[utf8]{inputenc}

% Define the page margin
\usepackage[margin=3cm]{geometry}

% Better typography (font rendering)
\usepackage{microtype}

% Math environments and macros
\usepackage{amsmath}
\usepackage{amsfonts}
\usepackage{amssymb}
\usepackage{amsthm}

% Define \includegraphics to include graphics
\usepackage{graphicx}

% Draw graphics from a text description
\usepackage{tikz}

% Syntax highlighting
\usepackage{minted}

% Set global minted options
\setminted{linenos, autogobble, frame=lines, framesep=2mm}

% Import the comment environment for orgtbl-mode
\usepackage{comment}

\title{Operating Systems, Sheet 4}
\author{Marten Lienen (03670270)}

\begin{document}

\maketitle

\section*{Exercise 1}

\subsection*{Part 1.1)}

\begin{itemize}
\item Ein Semaphor kontrolliert, dass nur einer gleichzeitig an der Rampe ist
\item Ein weiterer Semaphor kontrolliert, dass S nur an die Rampe fährt, wenn ein Schrank verfügbar ist
\item Ein letzter tut das Analoge für C
\end{itemize}

\subsection*{Part 1.2)}

\begin{minted}{c}
  rampe(1);
  schrank(0);
  couch(0);

  Process S {
    while (true) {
      P(schrank);

      P(rampe);
      V(rampe);
    }
  }

  Process C {
    while (true) {
      P(couch);

      P(rampe);
      V(rampe);
    }
  }

  Process F {
    while (true) {
      P(rampe);

      V(couch);
      V(schrank);

      V(rampe);
    }
  }
\end{minted}

\section*{Exercise 2}

Nein, weil vor dem Schalten einer Transition gleichzeitig gelten muss, dass alle Eingaben ausreichend markiert sind \emph{und} alle nachfolgenden Stellen ausreichend Kapazität haben.
Dies ist hier nicht der Fall, weil $s1$ zwar ausreichend markiert ist, aber nicht genügend freie Kapazität hat ($0$).

\section*{Exercise 3}

\subsection*{Part 3.1)}

\subsection*{Part 3.2)}

\subsection*{Part 3.3)}

\subsection*{Part 3.4)}

\section*{Exercise 4}

\subsection*{Part 4.1)}

\begin{description}
\item[Race Condition] Ein Bereich ist nicht ausreichend geschützt, sodass das Ergebnis davon abhängt, welcher Prozess den Bereich zuerst betritt
\item[Mutual Exclusion] Es kann jeweils nur ein Prozess den Wert der globalen Struktur ändern
\item[Critical Section] Ein Bereich des Programms vor dessen Eintritt die globale Struktur einen Freiwert haben und auf einen Besetztwert gesetzt werden muss
\item[Busy Waiting] Abfragen der globalen Struktur, bis sie den Freiwert annimmt
\item[Deadlock] Zwei Prozesse warten darauf, dass der jeweils andere Prozess einen globalen Wert auf den Freiwert setzt, auf den man selbst wartet
\end{description}

\subsection*{Part 4.2)}

In Java gibt es den Monitor, bzw. ist jedes Objekt ein Monitor.
Ein Monitor ist quasi ein binärer Semaphor, wobei $wait$ $P$ ist und $notify$ $V$.

\subsection*{Part 4.3)}

\subsubsection{Frage 1}

Das Problem ist, dass die blockierten Prozesse ihre Rechenzeit nicht direkt wieder abgeben können, sondern stattdesssen immer wieder das Schloss abfragen, bis sie vom Betriebssystem unterbrochen werden.

\subsubsection{Frage 2}

Der Speicher, in dem das Schloss liegt, muss von allen Prozessoren geteilt werden.
Falls es im Hauptspeicher liegt, müssen alle prozessornahen Caches aktualisiert werden.

\subsubsection{Frage 3}

Man benötigt eine atomare Compare-And-Set-Operation.

\section*{Exercise 5}

\subsection*{Part 5.1)}

Es gibt 4 Prozesse, jeweils einen für jedes Auto.
Es gibt 4 Resourcen, jeweils eine für die Vorfahrt vor einem anderen Auto.
Jedes Auto hat Vorfahrt vor jeweils einem anderen, also hält jeder Prozess eine dieser Resourcen.
Um fahren zu dürfen, ist jedoch die Vorfahrt vor dem linken und rechten Auto notwendig.
Also fordert jedes Auto noch eine weitere Resource, die aber schon von einem anderen Auto belegt wird.

\begin{enumerate}
\item Es können nicht zwei Autos Vorfahrt voreinander haben
\item Das Fahren auf der Kreuzung kann nicht unterbrochen werden
\item Jeder Prozess hat per Gesetz automatisch Vorfahrt vor dem rechten Auto, auch während darauf gewartet wird, dass links frei wird
\item Alle Autos warten zirkulär aufeinander
\end{enumerate}

Ja, es gibt einen Zykel einmal um die Kreuzung herum.

\subsection*{Part 5.2)}

Man könnte die jeweils gegenüberliegenden Vorfahrten mit nur einem Lock schützen, sodass sie nur gemeinsam belegt werden können.

Mir fallen keine guten Möglichkeiten ein, die das Problem nicht nur woandershin verschieben.

\end{document}
