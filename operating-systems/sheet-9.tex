\documentclass[10pt,a4paper]{article}
\usepackage[utf8]{inputenc}

% Define the page margin
\usepackage[margin=3cm]{geometry}

% Better typography (font rendering)
\usepackage{microtype}

% Math environments and macros
\usepackage{amsmath}
\usepackage{amsfonts}
\usepackage{amssymb}
\usepackage{amsthm}

% Define \includegraphics to include graphics
\usepackage{graphicx}

% Draw graphics from a text description
\usepackage{tikz}

% Syntax highlighting
\usepackage{minted}

% Set global minted options
\setminted{linenos, autogobble, frame=lines, framesep=2mm}

% Import the comment environment for orgtbl-mode
\usepackage{comment}

% Do not indent paragraphs
\usepackage{parskip}

\title{Operating Systems, Sheet 9}
\author{Marten Lienen (03670270)}

\begin{document}

\maketitle

\section*{Exercise 1}

\subsection*{Part 1.1)}

Der physikalische Speicher wird komplett genutzt, wenn er vollständig mit Seiten gefüllt werden kann.
Die ist genau der Fall, wenn die Größe des Speichers durch die Größe der Seite teilbar ist.
Da $256 = 2^{8}$, sind dies also genau die $2$er-Potenzen bis $2^{8}$.

\begin{itemize}
\item 1
\item 2
\item 4
\item 8
\item 16
\item 32
\item 64
\item 128
\item 256
\end{itemize}

\subsection*{Part 1.2)}

Es gibt $2^{7} = 128$ Seiten, von denen $8 = 2^{3}$ gleichzeitig im Speicher sein können.
Um das System zu betreiben, braucht man mindestens ausreichend Speicher, um die Seitentabelle speichern zu können.
Wenn wir eine einebige Seitentabelle annehmen, ist diese $128 \cdot K$ wobei $K$ die Größe eines Seitendeskriptors ist.
Wenn man dazu noch möchte, dass theoretisch der gesamte virtuelle Adressraum benutzt kann, braucht man zusätzlich noch $2^{12} - 8 \cdot 2^{5} + 2^{5}$ Byte, damit man alle Seiten auslagern kann, die nicht in den Speicher passen plus eine Seite zum Zwischenspeichern, während des Austauschens von Seiten.

\subsection*{Part 1.3)}

Da $32 = 2^{5}$, braucht man $5$ Bit, um alle Adressen innerhalb einer Seite adressieren zu können.
Dementsprechend entfallen die restlichen $7$ Bit auf die Seitennummer.

\section*{Exercise 2}

\subsection*{Part 2.1)}

% BEGIN RECEIVE ORGTBL exercise-2-1
\begin{tabular}{llll}
Adresse / Seite & 4KB & 8KB & 16KB\\
\hline
16 Bit & $2^{4}$ & $2^{3}$ & $2^{2}$\\
32 Bit & $2^{20}$ & $2^{19}$ & $2^{18}$\\
64 Bit & $2^{52}$ & $2^{51}$ & $2^{50}$\\
\end{tabular}
% END RECEIVE ORGTBL exercise-2-1
\begin{comment}
#+ORGTBL: SEND exercise-2-1 orgtbl-to-latex :splice nil :skip 0 :raw t
| Adresse / Seite | 4KB      | 8KB      | 16KB     |
|-----------------+----------+----------+----------|
| 16 Bit          | $2^{4}$  | $2^{3}$  | $2^{2}$  |
| 32 Bit          | $2^{20}$ | $2^{19}$ | $2^{18}$ |
| 64 Bit          | $2^{52}$ | $2^{51}$ | $2^{50}$ |
\end{comment}

\subsection*{Part 2.2)}

% BEGIN RECEIVE ORGTBL exercise-2-2
\begin{tabular}{lrrrr}
Seitengröße & Bits Seitennummer & Bits Seitenoffset & Bits Kachelnummber & Bits Kacheloffset\\
\hline
1 KB & 22 & 10 & 14 & 10\\
2 KB & 21 & 11 & 13 & 11\\
4 KB & 20 & 12 & 12 & 12\\
8 KB & 19 & 13 & 11 & 13\\
\end{tabular}
% END RECEIVE ORGTBL exercise-2-2
\begin{comment}
#+ORGTBL: SEND exercise-2-2 orgtbl-to-latex :splice nil :skip 0 :raw t
| Seitengröße | Bits Seitennummer | Bits Seitenoffset | Bits Kachelnummber | Bits Kacheloffset |
|-------------+-------------------+-------------------+--------------------+-------------------|
| 1 KB        |                22 |                10 |                 14 |                10 |
| 2 KB        |                21 |                11 |                 13 |                11 |
| 4 KB        |                20 |                12 |                 12 |                12 |
| 8 KB        |                19 |                13 |                 11 |                13 |
\end{comment}

\subsection*{Part 2.3)}

Man muss so viele Bits wählen, dass jede Kachel eindeutig adressiert werden kann.
Wenn man die ersten Bits nutzt, werden alle Seiten, deren Adressen die ersten $k$ Bits teilen, in die selbe Kachel geladen.
Analog für die mittleren und letzen Bits der Seitennummer.
Da Programme generell eine räumliche Lokalität an den Tag legen, würden sich sehr viele Seitennummern die ersten Bits teilen, wenn man die ersten Bits nutzen würde.
Der Großteil des Programms würde wohl auf dieselbe Kachel abgebildet werden, was zu vielen Seitenfehlern führen würde.
Deshalb empfiehlt es sich, die letzten Bits zu nutzen, um die Seiten in auf möglichst viele Kacheln zu verteilen.

\section*{Exercise 3}

\subsection*{Part 3.1)}

\begin{description}
\item[Blockoffset] Offset innerhalb eines Cache-Blocks (hier 2 Bit)
\item[Mengenindex] Index des Sets in welchem ein Block gecacht wird (hier 3 Bit)
\item[Tag] Eindeutiger Identifier unter allen Reihen, die in das selbe Set gecacht werden (hier 8 Bit)
\end{description}

\subsection*{Part 3.2)}

\begin{equation*}
  0E34_{16} = 0111000110100_{2}
\end{equation*}
\begin{equation*}
  CO = 0 \qquad CT = 10001101_{2} = AD_{16} \qquad CI = 011_{2} = 3_{10}
\end{equation*}

Es gibt einen Cache-Miss, weil in Menge $3$ keine Reihe mit Tag $AD$ gespeichert ist.
Der zurückgegebene Wert ist der, der an der entsprechenden Stelle im Speicher gelesen wird, nachdem die Reihe in den Cache geladen wurde.

\subsection*{Part 3.3)}

\begin{equation*}
  0DD5_{16} = 0110111010101_{2}
\end{equation*}
\begin{equation*}
  CI = 011_{2} = 3 \qquad CT = 01110101_{2} = 76_{16} \qquad CT = 01_{2} = 1_{10}
\end{equation*}

Es gibt wieder einen Cache-Miss.

\subsection*{Part 3.4)}

\subsubsection*{Frage 1}

Mit Lokalität meint man, dass Programme meistens Befehle ausführen und auf Daten zugreifen, die in bestimmter Hinsicht nah zusammen liegen.

\begin{description}
\item[Räumliche Lokalität] Daten und Programmteile, die der Prozessor braucht, sind in nah zusammenliegenden Adressen gespeichert
\item[Zeitliche Lokalität] Zugriffe auf einen Daten- oder Programmabschnitt liegen zeitlich nah zusammen
\end{description}

\subsubsection*{Frage 2}

\emph{Aliasing} bedeutet, dass die selben physischen Speicherzellen an verschiedenen Orten im Cache liegen können, was zu Konsistenzproblemen führen kann.

\subsubsection*{Frage 3}

Der \emph{Critical Stride} ist die Gesamtgröße des Caches.
Dies heißt so, weil Adressen, die ein Vielfaches vom CS auseinander liegen auf dieselbe Cache-Zeile abgebildet werden.
In ausreichend großen Arrays kann dies vorkommen und bedeutet, dass Cache-Zeilen sehr oft neu geladen werden müssen.

\section*{Exercise 4}

\subsection*{Part 4.1)}

Es bleiben $32 - 9 - 11 = 14$ Bit für den Seitenoffset.
Also sind die Seiten $2^{14}$ Byte bzw. 16 KB groß.

\subsection*{Part 4.2)}

Es gibt $2^{32 - 14} = 2^{9 + 11} = 2^{18}$ Seiten.

\subsection*{Part 4.3)}

Die obere Seitentabelle wäre $2^{9} \cdot 4 = 2^{11}$ Byte bzw. 8 KB groß.
Wenn dann jede der $2^{9}$ Unterseitentabellen realisiert ist, kommen dazu nochmal $2^{9} \cdot 2^{11} \cdot 4 = 2^{20}$ Byte bzw. 1 MB.

\section*{Exercise 5}

\inputminted{c}{sheet-9/main.c}

\begin{minted}{shell}
  # ./main
  Elapsed time creating a random matrix: 0.241544 seconds
  row-wise 0.059603 seconds -> 8390022.909666
  column-wise 0.207783 seconds -> 8390022.909670
  random 0.087708 seconds -> 2479967.500000
\end{minted}

Row-wise is the fastest, column-wise the slowest and random is somewhere in between.

\end{document}
