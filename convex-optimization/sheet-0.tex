\documentclass[10pt,a4paper]{article}
\usepackage[utf8]{inputenc}

% Define the page margin
\usepackage[margin=3cm]{geometry}

% Better typography (font rendering)
\usepackage{microtype}

% Math environments and macros
\usepackage{amsmath}
\usepackage{amsfonts}
\usepackage{amssymb}
\usepackage{amsthm}

% Define \includegraphics to include graphics
\usepackage{graphicx}

% Syntax highlighting
\usepackage{minted}

% Set global minted options
\usepackage{color}
\setminted{linenos, autogobble, mathescape, numbersep=5pt}

% Import the comment environment for orgtbl-mode
\usepackage{comment}

% Do not indent paragraphs
\usepackage{parskip}

\DeclareMathOperator{\tovec}{vec}

\title{Convex Optimization for Computer Vision, Sheet 0}
\author{Marten Lienen (03670270)}

\begin{document}

\maketitle

\section*{Exercise 1}

$D_{x}$ and $D_{y}$ are $n \times n$ and $m \times m$ diagonal matrices with $1$ on the main diagonal and $-1$ on the first sub- respectively superdiagonal.

\section*{Exercise 2}

\begin{minted}{matlab}
  f = double(rgb2gray(imread('Vegetation-028.jpg')));
  nm = size(f);
  n = nm(1);
  m = nm(2);
  ex = double(ones(m, 1));
  ey = double(ones(n, 1));
  Dx = spdiags([-1 * ex, ex], [0, -1], m, m);
  Dy = spdiags([-1 * ey, ey], [0, 1], n, n);
  imshow(f * Dx, [-255, 255]);
  imshow(Dy * f, [-255, 255]);
\end{minted}

\section*{Exercise 3}

\begin{equation*}
  I_{m} \cdot f \cdot \widetilde{D}_{x} = \nabla_{x} f \Leftrightarrow (\widetilde{D}_{x}^{T} \otimes I_{m}) \cdot \tovec(f) = \tovec(\nabla_{x} f) \Rightarrow D_{x} = \widetilde{D}_{x}^{T} \otimes I_{m}
\end{equation*}
\begin{equation*}
  \widetilde{D}_{y} \cdot f \cdot I_{n} = \nabla_{y} f \Leftrightarrow (I_{n} \otimes \widetilde{D}_{y}) \cdot \tovec(f) = \tovec(\nabla_{y} f) \Rightarrow D_{y} = I_{n} \otimes \widetilde{D}_{y}
\end{equation*}

\begin{minted}{matlab}
  F = f(:);
  E = ones(n * m, 1);
  DX = spdiags([-1 * E, E], [0, n], n * m, n * m);
  DY = spdiags([-1 * E, repmat([0; ones(n - 1, 1)], m, 1)], [0, 1], n * m, n * m);
  norm(DX * F - reshape(f * Dx, n * m, 1)) % => 0
  norm(DY * F - reshape(Dy * f, n * m, 1)) % => 0

  DX = kron(Dx', eye(n));
  DY = kron(eye(m), Dy);
  norm(DX * F - reshape(f * Dx, n * m, 1)) % => 0
  norm(DY * F - reshape(Dy * f, n * m, 1)) % => 0
\end{minted}

\section*{Exercise 4}

\begin{minted}{matlab}
  f3 = double(imread('Vegetation-028.jpg'));
  F3 = f3(:);
  nm = size(f3);
  n = nm(1);
  m = nm(2);
  en = ones(n, 1);
  em = ones(m, 1);
  Dx = spdiags([-1 * em, em], [0, -1], m, m);
  Dy = spdiags([-1 * en, en], [0, 1], n, n);
  DX = kron(Dx', speye(n));
  DY = kron(speye(m), Dy);
  nablac = cat(1, kron(eye(3), DX), kron(eye(3), DY));
\end{minted}

\section*{Exercise 5}

\begin{minted}{matlab}
  deriv = reshape(nablac * F3, n * m, 6);
  TV = sum(sqrt(sum(deriv.^2, 2)));
\end{minted}

The total variation of the second image is almost an order of magnitude larger than the total variation of the first image ($6 \cdot 10^{6}$ vs. $4 \cdot 10^{7}$).
I assume that the total variation of the first image is smaller because most of the image is a monotone, flat blue and thus small gradient.

\end{document}
